%%%%%%%%%%%%%%%%%%%%%%%%%%%%%%%%%%%%%%%%%%%%%%%%%%%%%%%%%%%%
% File: hw.tex 						   %
% Description: LaTeX template for homework.                %
%
% Feel free to modify it (mainly the 'preamble' file).     %
% Contact hfwei@nju.edu.cn (Hengfeng Wei) for suggestions. %
%%%%%%%%%%%%%%%%%%%%%%%%%%%%%%%%%%%%%%%%%%%%%%%%%%%%%%%%%%%%

%%%%%%%%%%%%%%%%%%%%%%%%%%%%%%%%%%%%%%%%%%%%%%%%%%%%%%%%%%%%%%%%%%%%%%
% IMPORTANT NOTE: Compile this file using 'XeLaTeX' (not 'PDFLaTeX') %
%
% If you are using TeXLive 2016 on windows,                          %
% you may need to check the following post:                          %
% https://tex.stackexchange.com/q/325278/23098                       %
%%%%%%%%%%%%%%%%%%%%%%%%%%%%%%%%%%%%%%%%%%%%%%%%%%%%%%%%%%%%%%%%%%%%%%

\documentclass[11pt, a4paper, UTF8]{ctexart}
\input{preamble}  % modify this file if necessary

%%%%%%%%%%%%%%%%%%%%
\title{数模调研}
\me{丁保荣}{171860509}
\date{\today}     % you can specify a date like ``2017年9月30日''.
%%%%%%%%%%%%%%%%%%%%
\begin{document}
\maketitle
\tableofcontents
\newpage
\section{数学建模是什么}
简单的来说:数学建模就是使用数学方法解决实际应用问题,在数学的框架下表达自己解决问题的思想和方法。\\

\subsection{数学建模的步骤}
\begin{enumerate}
\item 分析实际问题中的各种因素,使用变量表示;
\item 分析这些变量之间的关系,哪些是相互依存的,哪些是独立的,他们具有什么样的关系;
\item 根据实际问题选用合适的数学框架(典型的有优化问题,配置问题等等),并具体的应用问题在这个数学框架下表出;
\item 选用合适的算法求解数学框架下表出的问题;
\item 使用计算结果解释实际问题,并且分析结果的可靠性
\end{enumerate}

\subsection{数学建模中需要的能力}
\begin{itemize}
\item 数学思维的能力;
\item 分析问题本质的能力;
\item 团队合作的能力;
\item 资料检索能力:Google等互联网资源,图书馆;
\item 编程的能力:常用的数学工具软件有MATLAB和Mathematica
\item 论文写作的能力
\end{itemize}

\section{常见数模竞赛及其要求}
\subsection{MCM/ICM(俗称美赛)}
MCM/ICM是Mathematical Contest in Modeling和Interdisciplinary Contest in Modeling的缩写,即“数学建模竞赛”和“交叉学科建模竞赛”。MCM始于1985年,ICM始于2000年,由COMAP(the Consortium for Mathematics and Its Application,美国数学及其应用联合会)主办,得到了SIAM,NSA,INFORMS等多个组织的赞助。MCM/ICM与其他著名数学竞赛(如Putnam数学竞赛)的区别在于其着重强调研究问题、解决方案的原创性、团队合作、交流以及结果的合理性。竞赛以三人(本科生)为一组,在四天时间内,就指定的问题完成从建立模型、求解、验证到论文撰写的全部工作。\\
\begin{itemize}
\item 一般在下半年可以开始报名(大约11月左右报名),Contests→Register for Contest(这里需要用指导老师的邮箱来注册,所以需要提前联系老师,确定老师愿意指导,用老师的邮箱号注册,每位老师最多指导2只队伍)。美赛报名费100美元,需要用VISA卡或者MASTER卡支付。
\item 比赛时间:春节前后
\item 论文提交:在网上提交,并且寄送纸质版到美国。
\item 无答辩
\item 奖状发放:大概4月左右网上自己下载获奖证书(大陆同学)
\item 美赛的奖项有:Outstanding Winner (1\%) 、Finalist(1\%)、Meritorous Winner(9\%)、Honoralbe Mention(31\%)、Successful Participant(57\%)。一般只要提交了文章至少能获得成功参赛奖。
\end{itemize}
\subsection{CUMCM(俗称国赛)}
全国大学生数学建模竞赛(CUMCM)创办于1992年,每年一届,目前已成为中国大陆高校规模最大的基础性学科竞赛,也是世界上规模最大的数学建模竞赛。2017年,近11万名大学生报名参加本项竞赛。\\
\begin{itemize}
\item 报名:报名时间可能每个大学不太一样,有的大学要先进行校赛预选,大约是在5-6月开始报名,报名请关注学校相关教务处网站、数学学院网站。报名费300元(有的学校会返还报名费来鼓励大家积极参与,获奖的话说不定学校还会给丰厚的奖金呢~~)。以团队报名,每个队伍不超过3人(所以也可以2人或者1人),每队须有一个指导教师。(关于组队的注意事项后面会详细讲到)
\item 培训:有的学校会在暑假小学期组织建模培训,如果有的话,建议可以去听听~没有培训的话,就自己好好看看呗~
\item 比赛时间:比赛一般在每年9月中上旬举行,比赛时间是从某个周五的上午8:00开始,为期三天三夜,截止到次周一上午8:00。(关于时间的分配我在后面也会详细讲讲)
\item 比赛期间:参赛队伍可以在比赛期间利用图书、互联网资料帮助建模,有问题也可以请教老师,原则上不相互交流(原则上......)。本科组比赛有A,B两道题,需要选择其中一道题进行解答。PS:最后AB两题各个奖项数量相同,所以如果选A,B题的分别有7000,3000只队伍,国赛一等奖A,B题分别有20个名额,那么A题的获奖比例和B题是不同的,但是具体选做的人少的还是选容易的要自己斟酌~(关于换题在后面会讲讲)
\item 比赛提交:提交纸质版给数学学院,并且把论文、数据、程序打包压缩拷贝给相关老师。
\item 比赛答辩:初审进入国赛获奖名单的队伍需要答辩,每个省的初审进度可能不太一样,有的在9月底就会进行答辩,有的可能10月。答辩开始有一个3-5分钟的概要介绍,每个队伍选一个口齿伶俐的小伙伴上去讲就好。答辩的主要目的是验真,所以只要是自己做的应该没多大问题。答辩可能会问到关于模型、软件或者程序的问题。当然答辩也是可能挂掉的,挂掉了就降档。
\item 奖项:国一  国二  省一  省二  省三
\end{itemize}

\section{准备}
\subsection{数学知识}
\begin{itemize}
\item 《线性代数》(很重要)
\item 《常微分方程》
\item 《运筹学》
\item 《微积分》(基础)
\item 《概率论与数理统计》(很重要)
\end{itemize}

\section{学校对于数模的安排}
\begin{itemize}
\item 需要数学和计算机两门学科最高一次考试成绩达到85(具体见报名通知时的细则)
\item 校内集训:暑假的头两(可能与小学期有冲突)。如果通过选拔,将持续培训到比赛前
\item 需要自行学习概率统计、线性代数、线性规划;数据分析软件等内容。
\item 推荐教材:叶其孝主编的《大学生数学建模竞赛辅导教材》系列共五本。在图书馆数量多极易借。
\end{itemize}




\section{数学建模的分工}
\subsection{建模}

\subsection{编程}
要达到熟练运用matlab进行运筹优化,数据处理,微分方程的地步. 数理统计可以交给SPSS,R ,其中SPSS无脑操作上手快.
\subsection{论文}
数学建模论文在20-25页上下最好,带上附页最好也别超过40页左右。

\section{推荐书目}
\begin{itemize}
\item 数学模型(姜启源、谢金星) (国产风格,面面俱到,模型比较杂,而且可操作性较差。适合作为一本辅助参考性读物进行学习)
\item 数学建模方法与分析.(新西兰)Mark.M.Meerschaert. (娓娓道来,五步法)
\item MATLAB揭秘  郑碧波 译 (本书讲的极其通俗易懂,适合无编程经验的)
\item 精通matlab2011a  张志涌
\item 数学建模与应用:司守奎 (囊括了各类建模的知识,还附有代码,很难得,工具书性质的)
\item Matlab智能算法30个案例分析  史峰,王辉等
\item 《MATLAB神经网络43个案例分析》- 王小川, 史峰, 郁磊    
\item 《MATLAB统计分析与应用:40个案例分析》
\item 数字图像处理(MATLAB版)  冈萨雷斯 (13国赛碎纸片复原居然涉及了图像处理,所以列在这里了.可看可不看,太专业化了)
\item 《数学建模竞赛:获奖论文精选与点评》- 韩中庚
\item 《机器学习》 - 周志华
\item 《统计学习方法》 - 李航
\item 《最优化理论与方法》 - 袁亚湘
\item 《最优化原理》 - 胡适耕
\item 《凸优化(中译)》 - Stephen Boyd
\item 《凸优化算法(英文)》 - Dimitri P.Bertsekas
\item 《Introduction to Numerical Analysis (英文)》- J.Stoer,R.Bulirsch
\item 《数据挖掘导论 (中译)》-Michael SteinBach
\item 《正确写作美国大学生数学建模竞赛》
\item 《美国大学生数学建模竞赛题解析与研究》1-5卷

\end{itemize}

\section{软件准备}
\begin{itemize}
\item 编程工具(Matlab/Python/Mathematica)
\item 统计建模(R/SPSS/Minitab)
\item 论文写作(Word/LaTex)
\item 公式编辑器(MathType)
\item 插图制作(PowerPoint/PS)
\item 流程图绘制(Visio)
\item 版本控制器(SVN/Git)
\item 团队资料笔记共享(有道云笔记)
\end{itemize}


\section{前人经验}
\begin{itemize}
\item 国赛讲究实力,美赛讲究创新。美赛不一定要多高级的方法,但是一定要有创意。而国赛,组委会往往是有一个模糊的“标准答案”在的,按部就班做下来就好了。拿到一个问题,可以先建立一个初等模型,讨论下结果;再逐渐放宽条件,把模型做的复杂一点。即 Basic model -> Normal model -> Extended model的思路。这个思维在美赛中很好,这么做下来基本都能得金奖的。
\item MATLAB是必备的,必须要熟练掌握各种模型的实现。此外,SPSS(或者R)也是要掌握的。Mathematic和MATLAB的替代性很强,不掌握也没关系(仅在建模方面,mathematic 当然也是很强大的)。强烈建议大家至少熟练掌握一种智能算法.
\item 摘要很重要
\item 硕博论文是一个非常好的突破口。
\end{itemize}

\newpage
%%%%%%%%%%%%%%%%%%%%
% always keep this
%%%%%%%%%%%%%%%%%%%%
% begin ``this homework (hw)'' part

\begin{thebibliography}{99}
\bibitem{a} wiki中关于数学建模的条目 https://zh.wikipedia.org/wiki/数学建模
\bibitem{b} 知乎上关于如何入门参与数学建模的回答 
\bibitem{c} 团组织生活-计科竞赛一览.pdf by郑奘巍
\end{thebibliography}
\end{document}