%%%%%%%%%%%%%%%%%%%%%%%%%%%%%%%%%%%%%%%%%%%%%%%%%%%%%%%%%%%%
% File: hw.tex 						   %
% Description: LaTeX template for homework.                %
%
% Feel free to modify it (mainly the 'preamble' file).     %
% Contact hfwei@nju.edu.cn (Hengfeng Wei) for suggestions. %
%%%%%%%%%%%%%%%%%%%%%%%%%%%%%%%%%%%%%%%%%%%%%%%%%%%%%%%%%%%%

%%%%%%%%%%%%%%%%%%%%%%%%%%%%%%%%%%%%%%%%%%%%%%%%%%%%%%%%%%%%%%%%%%%%%%
% IMPORTANT NOTE: Compile this file using 'XeLaTeX' (not 'PDFLaTeX') %
%
% If you are using TeXLive 2016 on windows,                          %
% you may need to check the following post:                          %
% https://tex.stackexchange.com/q/325278/23098                       %
%%%%%%%%%%%%%%%%%%%%%%%%%%%%%%%%%%%%%%%%%%%%%%%%%%%%%%%%%%%%%%%%%%%%%%

\documentclass[11pt, a4paper, UTF8]{ctexart}
%%%%%%%%%%%%%%%%%%%%%%%%%%%%%%%%%%%
% File: preamble.tex
%%%%%%%%%%%%%%%%%%%%%%%%%%%%%%%%%%%

\usepackage[top = 1.5cm]{geometry}

% Set fonts commands
\newcommand{\song}{\CJKfamily{song}} 
\newcommand{\hei}{\CJKfamily{hei}} 
\newcommand{\kai}{\CJKfamily{kai}} 
\newcommand{\fs}{\CJKfamily{fs}}

\newcommand{\me}[2]{\author{{\bfseries 姓名:}\underline{#1}\hspace{2em}{\bfseries 学号:}\underline{#2}}}

% Always keep this.
\newcommand{\noplagiarism}{
  \begin{center}
    \fbox{\begin{tabular}{@{}c@{}}
      请独立完成作业,不得抄袭。\\
      若参考了其它资料,请给出引用。\\
      鼓励讨论,但需独立书写解题过程。
    \end{tabular}}
  \end{center}
}

% Each hw consists of three parts:
% (1) this homework
\newcommand{\beginthishw}{\part{作业}}
% (2) corrections (Optional)
\newcommand{\begincorrection}{\part{订正}}
% (3) any feedback (Optional)
\newcommand{\beginfb}{\part{反馈}}

% For math
\usepackage{amsmath}
\usepackage{amsfonts}
\usepackage{amssymb}

% Define theorem-like environments
\usepackage[amsmath, thmmarks]{ntheorem}

\theoremstyle{break}
\theorembodyfont{\song}
\theoremseparator{}
\newtheorem*{problem}{题目}

\theorempreskip{2.0\topsep}
\theoremheaderfont{\kai\bfseries}
\theoremseparator{:}
% \newtheorem*{remark}{注}
\theorempostwork{\bigskip\hrule}
\newtheorem*{solution}{解答}
\theorempostwork{\bigskip\hrule}
\newtheorem*{revision}{订正}

\theoremstyle{plain}
\newtheorem*{cause}{错因分析}
\newtheorem*{remark}{注}

\theoremstyle{break}
\theorempostwork{\bigskip\hrule}
\theoremsymbol{\ensuremath{\Box}}
\newtheorem*{proof}{证明}

\renewcommand\figurename{图}
\renewcommand\tablename{表}

% For figures
% for fig with caption: #1: width/size; #2: fig file; #3: fig caption
\newcommand{\fig}[3]{
  \begin{figure}[htp]
    \centering
      \includegraphics[#1]{#2}
      \caption{#3}
  \end{figure}
}

% for fig without caption: #1: width/size; #2: fig file
\newcommand{\fignocaption}[2]{
  \begin{figure}[htp]
    \centering
    \includegraphics[#1]{#2}
  \end{figure}
}  % modify this file if necessary

%%%%%%%%%%%%%%%%%%%%
\title{第十一讲:有限与无限}
\me{丁保荣}{171860509}
\date{\today}     % you can specify a date like ``2017年9月30日''.
%%%%%%%%%%%%%%%%%%%%
\begin{document}
\maketitle
%%%%%%%%%%%%%%%%%%%%
\noplagiarism	% always keep this
%%%%%%%%%%%%%%%%%%%%
\beginthishw	% begin ``this homework (hw)'' part

%%%%%%%%%%
\begin{problem}[UD:20.4]
(a) Show that the positive rationals $\mathbb{Q}^+$ and the negative rationals$\mathbb{Q}^-$ are equivalent.\\
(b) Show that the even and odd integers are equivalent.
\end{problem}
\begin{solution}
(a) Define $f$: $\mathbb{Q}^+$ $\rightarrow$ $\mathbb{Q}^-$ by f(x)=-x\\
Firstly, we can easily note that it is well-defined.\\
Then, for every y$\in$ $\mathbb{Q}^-$, there is one and only one x $\in$ $\mathbb{Q}^+$ that satisfies f(x)=-x\\
Therefore f is bijective.\\
$\therefore$  the positive rationals $\mathbb{Q}^+$ and the negative rationals$\mathbb{Q}^-$ are equivalent.\\
(b) Define $f$: \{2k|k$\in$Z\} $\rightarrow$ \{2k-1|k$\in$Z\} by f(x)=x+1\\
Firstly, we can easily note that it is well-defined.\\
Then, for every y$\in$ $\mathbb{Q}^-$, there is one and only one x $\in$ $\mathbb{Q}^+$ that satisfies f(x)=-x\\
Therefore f is bijective.\\
$\therefore$ the even and odd integers are equivalent.
\end{solution}

\begin{problem}[UD:20.8]
Prove Theorem 20.6 working with the outline given in the text.
\end{problem}
\begin{solution}
Firstly, for every x$\in$ A$\cup$B, x$\in$ A or $\in$B (but not both), therefore there exists only one y that satisfies H(x)=y for every x$\in$ A$\cup$B\\
$\therefore$ H(x) is well-defined.\\
Then, for every y$\in$ C$\cup$D, y$\in$C or $\in$D (but not both). We just illusrate the case that y$\in$C, for y$\in$D, it is similar\\
$\because$ f is bijective and A$\cap$B=$\emptyset$\\
$\therefore$ for every y$\in$C, there is one and only one x$\in$A that satisfies f(x)=y and there is no x$\in$B that satisfies f(x)=y.\\
$\therefore$ H(x) is bijective.\\
$\therefore$ H(x) is well-defined and bijective.\\
$\therefore$ A$\cup$B $\approx$ C$\cup$D.
 \end{solution}


\begin{problem}[UD:20.9]
(a) Suppose that A and B are nonempty finite sets and A$\cap$B=$\emptyset$. Show that there exist integers $n$ and $m$ such that $A$ $\approx$ \{1,2,$\dots$,$n$\} and $B$ $\approx$ \{$n+1$,$\dots$,$n+m$\}.\\
(b) Prove Corollary 20.8.
\end{problem}
\begin{solution}
(a) Since A is a nonempty finite set, we can define f: A$\rightarrow$\{1,2,3,$\dots$,n\} for some n$in$N\\
$\therefore$ A$\approx$\{1,2,3,$\dots,n$\} for some n$\in$N\\
Similarly, we can define g: B$\rightarrow$ \{1,2,3,$\dots$,m\} for some m$\in$N\\
Then we define a bijective function h: \{1,2,3,$\dots,m$\} $\rightarrow$ \{n+1,n+2,$\dots$,n+m\} by f(x)=x+n\\
By Theorem 15.6, the comosition h$\circ$g: B$\rightarrow$ \{n+1,n+2,$\dots$,n+m\} for some n and m $\in$N  is bijective.\\
$\therefore$ $B$ $\approx$ \{$n+1$,$\dots$,$n+m$\}.\\
(b) 
let C=\{1,2,3,$\dots$,n\}, D=\{n+1,n+2,$\dots$,n+m\}  Therefore C$\cap$D=$\emptyset$\\
Define 
\[H(x)= \begin{cases}
f(x) & \text{if } x\in A\\
g(x) & \text{if } x\in B
\end{cases}\]
By Theorem 20.6, we can have A$\cup$B is finite.

\end{solution}


\begin{problem}[UD:20.10]
Prove Theorem 20.14 below. We suggest that you start by working Problem 15.14 if you have not already done so.\\
$\boldsymbol{Theorem 20.14.}$\\
Let A,B,C and D be nonempty sets with A$\approx$C and B$\approx$D. Then A$\times$B $\approx$ C$\times$D.
\end{problem}
\begin{solution}
$\because$ A$\approx$B and C$\approx$D\\
$\therefore$ there exist a bijective function f: A$\rightarrow$B and a bijective function g: C$\rightarrow$D\\
Define H: A$\times$C $\rightarrow$ B$\times$D by H(a,c)=(f(a),g(c)).\\
According to Problem 15.14, we can conclude that H is bijective.\\
$\therefore$ A$\times$B $\approx$ C$\times$D

\end{solution}


\begin{problem}[UD:21.7]
Show that $\mathbb{Q}$ is infinite.
\end{problem}
\begin{solution}
We know that N$\subsetneq$Z, adn Theorem 21.3 tells us that N is infinite. Since Corollary 20.11 says that every subset of a finite set is finite, or set Q must be infinite.
\end{solution}



\begin{problem}[UD:21.9]
Let A be a set, and suppose that B is an infinite subset of A. Show that A must be infinite.
\end{problem}
\begin{solution}
If A is finite, by Corollary 20.11, we have B is finite. It contradicts with that B is infinite.\\
Therefore A must be infinite.
\end{solution}



\begin{problem}[UD:21.10]
Suppose that A is an infinite set, B is a finite set and $f$: A$\rightarrow$B is a function. Show that there exists $b\in$B such that $f$$^{-1}$(\{$b$\}) is infinite.
\end{problem}
\begin{solution}
Let we suppose for all b$\in$B , we have $f$$^{-1}$(\{$b$\}) is finite.\\
By Theorem 20.12, we have $\bigcup_{b\in B}$$f$$^{-1}$(\{$b$\}) is finite.\\
And because $\bigcup_{b\in B}$$f$$^{-1}$(\{$b$\})=A\\
Therefore A is finite, but it contradcits with that A is infinite.\\
Therefore what we suppose is wrong.\\
Therefore  there exists $b\in$B such that $f$$^{-1}$(\{$b$\}) is infinite.
\end{solution}



\begin{problem}[UD:21.11]
Let $X$ be an infinite set, and $A$ and $B$ be finite subsets of $X$. Answer each of the following, giving reasons for your answers:\\
(a) Is $A\cap B$ finite or infinite?\\
(b) Is A$\backslash$B finite or infinite?\\
(c) Is X$\backslash$A finite or infinite?\\
(d) Is A$\cup$B finite or infinite?\\
(e) If $f$: A$\rightarrow$X is a one-to-one function, is $f(A)$ finite or infinite?
\end{problem}
\begin{solution}
(a) A$\cap$B is finite.\\
$\because$ A and B are finite\\
$\therefore$ A$\therefore$ A$\cup$B is finite.\\
$\because$ A$\cap$B $\subseteq$ A$\cup$B\\
$\therefore$ A$\cap$B is finite.\\
(b) A$\backslash$B is finite.\\
$\because$ A is finite\\
$\because$ A$\backslash$B $\subseteq$ A\\
$\therefore$ A$\backslash$B is finite.\\
(c) X$\backslash$A is infinite\\
Let we suppose  X$\backslash$A is finite\\
$\because$ A is finite\\
$\therefore$ X=A$\cup$ (X$\backslash$A) is finite, but it contradicts with that X is infinite.\\
$\therefore$ what we suppose is wrong.\\
$\therefore$ X$\backslash$A is infinite\\
(d) A$\cup$B is finite.\\
By Theorem 20.12, we can conclude that.\\
(e) f(A) is finite.\\
$\because$ A is one-to-one.\\
$\therefore$ A$\rightarrow$ f(A) is bijective\\
By Theorem 21.6, there is a unique positive integer n such that A$\approx$\{1,$\dots$,n\}.\\
$\therefore$ f(A) $\approx$ \{1,$\dots,n$\}\\
$\therefore$ f(A) is finite. 
\end{solution}



\begin{problem}[UD:21.16]
(a) Suppose that A is a finite set and B$\subseteq$A. We showed that B is finite. Show that |B|$\le$|A|.\\
(b) Suppose that A is a finite set and B$\subseteq$A. Show that if B$\not=$A, then |B|<|A|.\\
(c) Show that if two finite sets A and B satisfy B$\subseteq$A and |A|$\le$|B|, then A=B.
\end{problem}
\begin{solution}
(a) A=B$\cup$(B$\backslash$A)\\
(i) if (B$\backslash$A)$\not=$$\emptyset$ According to Problem 20.9, we can have B$\approx$\{1,2,$\dots$,n\} , (B$\backslash$A)=\{n+1,$\dots$,n+m\} and A=\{1,$\dots$,n+m\}\\
$\therefore$ |A|=|B|+|(B$\backslash$A)| \\
(ii) if (B$\backslash$A)=$\emptyset$, then A=B and |(B$\backslash$A)|=0\\
$\therefore$ |A|=|B|+|(B$\backslash$A)| \\
$\because$ |(B$\backslash$A)| $\ge$ 0\\
$\therefore$ |B|$\le$|A|.\\
(b)A=B$\cup$(B$\backslash$A)\\
$\because$ B$\not=$A\\
$\therefore$ (B$\backslash$A)$\not=$$\emptyset$\\
$\therefore$ According to Problem 20.9, we can have B$\approx$\{1,2,$\dots$,n\} , (B$\backslash$A)=\{n+1,$\dots$,n+m\} and A=\{1,$\dots$,n+m\}\\
$\therefore$ |A|=|B|+|(B$\backslash$A)| \\
$\because$ |(B$\backslash$A)| > 0\\
$\therefore$ |B|<|A|.\\
(c)\\
$\because$ B satisfy B$\subseteq$A\\
$\therefore$ |B|$\le$|A| and A=B$\cup$(B$\backslash$A)\\\\
$\because$ |A|$\le$|B|\\
$\therefore$ |A|=|B|\\
$\therefore$ |(B$\backslash$A)| = 0\\
$\therefore$ (B$\backslash$A)=$\emptyset$\\
$\therefore$ A=B
\end{solution}

\begin{problem}[UD:21.17]
Suppose that A and B are finite sets and $f$: A$\rightarrow$B is one-to-one. Show that |A|$\le$|B|.
\end{problem}
\begin{solution}
$\because$ A and B are finite sets\\
$\therefore$ \{1,$\dots,n$\}$\approx$A for some n$\in$N and B$\approx$\{1,$\dots,m$\} for some m$\in$N\\
$\therefore$ Define g: \{1,$\dots,n$\}$\rightarrow$A is bijective and h: B$\rightarrow$\{1,$\dots,m$\} is bijective\\
and $\because$ f is one-to-one,\\
$\therefore$ h$\circ$f$\circ$g: \{1,$\dots,n$\}$\rightarrow$ \{1,$\dots,m$\} is one-to-one.\\
$\therefore$ we can easily conclude that \{1,$\dots,n$\} $\subseteq$ \{1,$\dots,m$\}\\
$\therefore$ n$\le$m\\
$\therefore$ |A|$\le$|B|.\\
\end{solution}



\begin{problem}[UD:21.18]
Let A and B be sets with A finite. Let $f$: A$\rightarrow$B. Prove that |ran($f$)| $\le$ |A|.
\end{problem}
\begin{solution}
Let we suppose |ran($f$)| > |A|\\
$\therefore$ there exists an x which relates to at least two different y's\\
$\therefore$ it contradicts with f is a well-defined function\\
$\therefore$ what we suppose is not right.\\
$\therefore$ |ran($f$)| $\le$ |A|.
\end{solution}



\begin{problem}[UD:21.19]
Let A be a finite set. Show that a function $f$: A$\rightarrow$A is one-to-one if and only if it is onto. Is this still true if A is infinite?
\end{problem}
\begin{solution}
(a) $f$: A$\rightarrow$A is one-to-one.\\
Let we suppose f is not onto.\\
$\therefore$ ran(f)$\subsetneq$A\\
$\therefore$ |ran(f)|<|A|\\
$\therefore$ we can note that g: A$\rightarrow$ran(f) is also one-to-one\\
$\therefore$ |A|$\le$|ran(f)|\\
$\therefore$ |A|$\le$|ran(f)|<|A|, but it contradicts\\
$\therefore$ what we suppose is not right\\
$\therefore$ f is onto\\
(b) $f$: A$\rightarrow$A is onto.\\
Let we suppose f is not one-to-one\\
$\therefore$ there exist one y$\in$A that there are at least two x's $\in$ A that satisfy f(x)=y, we define them x$_1$ and x$_2$\\
$\therefore$ we define F: A$\backslash$\{x$_1$\} is still onto.\\
$\therefore$ |A| $\le$ |A$\backslash$\{x$_1$\}| =|A| -1,\\
$\therefore$ it contradicts\\
$\therefore$ what we suppose is wrong\\
$\therefore$ f is one-to-one\\
$\therefore$ $f$: A$\rightarrow$A is one-to-one if and only if it is onto.
\\
(c) it is not true if A is infinite.\\
\end{solution}




\begin{problem}[UD:22.1]
Give an example, if possible, of each of the following:\\
(a) a countably infinite collection of pairwise disjoint finite sets whose union is countably infinite;(See Problem 8.11 for the definition of pairwise disjoint.)\\
(b) a countably infinite collection of nonempty sets whose union is finite;\\
(c) a countably infinite collection of pairwise disjoint nonempty sets whose union is finite.\\
\end{problem}
\begin{solution}
(a) $\bigcup_{I\in N}$\{I\}\\
(b) not exist\\
(c) not exist\\
\end{solution}



\begin{problem}[UD:22.2]
Which of the following sets are finite? countably infinite? uncountable? (Be careful-don't apply theorems for finite sets to infinite sets!) Give reasons for your answers for each of the following:\\
(a) \{1/n: n$\in$ $\mathbb{Z}$$\backslash$\{0\}\};\\
(b) $\mathbb{R}$ $\backslash$ $\mathbb{N}$;\\
(c) \{$x\in \mathbb{Z}$:|$x$-7|<|$x$|\};\\
(d) 2$\mathbb{Z}$ $\times$ 3$\mathbb{Z}$;\\
(e) the set of all lines with rational slopes;\\
(f) $\mathbb{Q}$ $\backslash$ \{0\};\\
(g) $\mathbb{N}$ $\backslash$ \{1,3\}.
\end{problem}
\begin{solution}
(a) countably infinite.\\
As $\mathbb{Q}$ is countably infinite and the asked set is a subset of it.\\
(b) uncountable, the proof is similar to Theorem 22.12.\\
(c) countably infinite. The set is \{4,5,6,$\dots$\} and we can easily prove that \{4,5,6,$\dots$\}$\approx$ N.\\
(d) countably infinite. By Corollary 22.10\\
(e) uncountable, let y=ax+b, b can be irrational number.\\
(f) countably infinite.\\
As $\mathbb{Q}$ is countably infinite and the asked set is a subset of it.\\
(g) countably infinite.\\
As $\mathbb{N}$ is countably infinite and the asked set is a subset of it.\\
\end{solution}




\begin{problem}[UD:22.3]
Is the set of all infinite sequences of 0's and 1's finite, countably infinite, or uncountable? Guess and then prove, please.
\end{problem}

\begin{solution}
We can easily note that this set is not finite.\\
Let us suppose this set is countably infinite. And we have shown that $\mathbb{Z}^+$ is countably infinite, there exists a bijective function f: $\mathbb{Z}^+$ $\rightarrow$ (0,111111$\dots$). We list the values of f using the sequence of 1 or 0\\
f(1) = a$_{11}$a$_{12}$a$_{13}$$\dots$\\
f(2) = a$_{21}$a$_{22}$a$_{23}$$\dots$\\
f(3) = a$_{31}$a$_{32}$a$_{33}$$\dots$\\
$\dots$\\
Since f is onto, each sequence of 0's and 1's appears in this list.\\
The odd thing is that we can construct a number b=f(1) = b$_{1}$b$_{2}$b$_{3}$$\dots$\\ that is not in the list.\\
if a$_{11}$=0, then b$_{1}$=1 else b$_{1}$=1, and similarly we relates b$_{n}$ with a$_{nn}$ by assigning a different value from a$_{nn}$ to b$_{n}$\\
Therefore we construct a b that is not in the list, it contradicts with the assertion that f is onto.\\
$\therefore$ what we suppose is not right\\
$\therefore$ this set is not countably infinite and isn't finite.\\
$\therefore$ this set is uncoountable.
\end{solution}



\begin{problem}[UD:22.6]
Prove Corollary 22.4.
\end{problem}
\begin{solution}
let A be the countable set. There are two cases.\\
(i) A is finite\\
According to Corollay 20.11, we can conclude that every subset of A is finite.\\
Therefore every subset of A is countable.\\
(ii) A is infinite.\\
Therefore A is countable infinite.\\
Therefore A$\approx$N\\
Therefore we can construct a bijective function f: N$\rightarrow$A\\
Select an arbitrary subset of A, name it B\\
Therefore B $\subseteq$A\\
if B is finite, we can easily note that B is countable.\\
if B is infinite, we can construct a bijective function g: N$\rightarrow$B\\
$\therefore$ B is countable\\
$\therefore$ Every subset of a countable set is countable.
\end{solution}


\begin{problem}[UD:22.9]
There is another way to show that $\mathbb{Q}$ is countable. Turn the outline below into a proof by describing the counting process. (Don't try to find a formula for the function.)
\end{problem}
\begin{solution}
We can count the elements of Q$^{+}$ in the way illustrated in the grapg below.\\
And we have proved that Q$^{-}$ $\approx$ Q$^{+}$. Then Q=Q$^{+}$ $\cup$ Q$^{-}$ $\cup$ \{0\} is finite and countable, so Q$\approx$N\\
    \fig{width = 0.70\textwidth}{UD22.9.JPG}{counting solution}
\end{solution}


\begin{problem}[UD:23.2]
(a) In $\mathbb{R}$, find the distance of the number 1 to the number 3 in the usual metric and in the discrete metric.\\
(b) In $\mathbb{R}$$^2$, find the distance of the point (1,3) to the point (2,5) in the usual metric, the taxicab metric, the max metric, and the discrete metric.
\end{problem}
\begin{solution}
(a) d$_{u}$(1,3)=2, d$_{d}$(1.3)=1\\
(b) d$_{u}$((1,3),(2,5))=$\sqrt{5}$\\
d$_{tc}$((1,3),(2,5))=3\\
d$_{m}$((1,3),(2,5))=2\\
d$_{d}$((1,3),(2,5))=1\\
\end{solution}



\begin{problem}[UD:23.3]
(a) Sketch the set \{$(x,y)\in \mathbb{R}^2$: $d_u((x,y),(0,0)<1)$\}, where $d_u$ is the usual metric.\\
(b) Sketch the set \{$(x,y)\in \mathbb{R}^2$: $d_{tc}((x,y),(0,0)<1)$\}, where $d_{tc}$ is the taxicab metric.\\
(c) Sketch the set \{$(x,y)\in \mathbb{R}^2$: $d_m((x,y),(0,0)<1)$\}, where $d_m$ is the max metric.\\
(d) Sketch the set \{$(x,y)\in \mathbb{R}^2$: $d_d((x,y),(0,0)<1)$\}, where $d_d$ is the discrete metric.\\
(e) Sketch the set \{$(x,y,z)\in \mathbb{R}^3$: $d_u((x,y,z),(0,0,0)<1)$\}, where $d_u$ is the usual metric. (See Example 23.2 for the definition if you need it.)\\
\end{problem}
\begin{solution}
\fig{width = 0.70\textwidth}{UD23.3.JPG}{sketching solution}
\end{solution}



\begin{problem}[UD:23.10]
Let $X$ be the space of polynomials with real coefficients. Define a function d from X$\times$X $\rightarrow$ $\mathbb{R}$ by $d(p,q)=|p(0),q(0)|.$ Is $d$ a metric? If so, prove it. If not,why not?
\end{problem}
\begin{solution}
No, it doesn't satisfy definiteness: let p=x, q=x$^2$, then |p(0)=q(0)|, but p$\not=$q.
\end{solution}




%%%%%%%%%%%%%%%%%%%%
\begincorrection	% begin the ``correction'' part (Optional)

%%%%%%%%%%
\begin{problem}[UD: 10.5]
  Let X be a nonempty set with an equivalence relation $\sim$ on it. Prove that for all elements x and y in X, the equality E$_x$=E$_y$ holds if and only if x$\sim$y.
\end{problem}

\begin{cause}
  简述错误原因(可选)。
\end{cause}

% Or use the ``solution'' environment
\begin{revision}
  (1)\\
  $\because$ [x]=[y]\\
  $\therefore$ x$\in$[x]=[y]\\
  $\therefore$ x$\sim$y\\
  (2)\\
  for a$\in$[x]\\
  $\therefore$a$\sim$x$\sim$y\\
  $\therefore$a$\sim$y\\
  $\therefore$a$\in$[y]\\
  $\therefore$ [x]$\subseteq$[y]\\
  Similarly, we can have [y]$\subseteq$[x]\\
  $\therefore$ [x]=[y]
\end{revision}
%%%%%%%%%%
%%%%%%%%%%%%%%%%%%%%
\beginfb	% begin the feedback section (Optional)

你可以写:
\begin{itemize}
  \item 对课程及教师的建议与意见
  \item 教材中不理解的内容
  \item 希望深入了解的内容
  \item 等
\end{itemize}
%%%%%%%%%%%%%%%%%%%%
\end{document}