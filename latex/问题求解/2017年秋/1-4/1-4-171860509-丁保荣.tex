%%%%%%%%%%%%%%%%%%%%%%%%%%%%%%%%%%%%%%%%%%%%%%%%%%%%%%%%%%%%
% File: hw.tex 						   %
% Description: LaTeX template for homework.                %
%
% Feel free to modify it (mainly the 'preamble' file).     %
% Contact hfwei@nju.edu.cn (Hengfeng Wei) for suggestions. %
%%%%%%%%%%%%%%%%%%%%%%%%%%%%%%%%%%%%%%%%%%%%%%%%%%%%%%%%%%%%

%%%%%%%%%%%%%%%%%%%%%%%%%%%%%%%%%%%%%%%%%%%%%%%%%%%%%%%%%%%%%%%%%%%%%%
% IMPORTANT NOTE: Compile this file using 'XeLaTeX' (not 'PDFLaTeX') %
%
% If you are using TeXLive 2016 on windows,                          %
% you may need to check the following post:                          %
% https://tex.stackexchange.com/q/325278/23098                       %
%%%%%%%%%%%%%%%%%%%%%%%%%%%%%%%%%%%%%%%%%%%%%%%%%%%%%%%%%%%%%%%%%%%%%%

\documentclass[11pt, a4paper, UTF8]{ctexart}
\input{preamble}  % modify this file if necessary
\usepackage{listings}
\lstloadlanguages{C++, csh, make}
\lstset{language=C++,tabsize=4, keepspaces=true,
	breakindent=22pt, 
	numbers=left,stepnumber=1,numberstyle=\tiny,
	basicstyle=\footnotesize, 
	showspaces=false,
	flexiblecolumns=true,
	breaklines=true, breakautoindent=true,breakindent=4em,
	escapeinside={/*@}{@*/}
}
%%%%%%%%%%%%%%%%%%%%
\title{第四讲:基本的算法结构}
\me{丁保荣}{171860509}
\date{\today}     % you can specify a date like ``2017年9月30日''.
%%%%%%%%%%%%%%%%%%%%
\begin{document}
\maketitle
%%%%%%%%%%%%%%%%%%%%
\noplagiarism	% always keep this
%%%%%%%%%%%%%%%%%%%%
\beginthishw	% begin ``this homework (hw)'' part

%%%%%%%%%%
\begin{problem}[DH:2.1]	% NOTE: use '[]' (instead of '()' or '{}') to provide additional information
   The algorithm for summing the salaries of N employees presented in the text performs a loop that consists of adding one salary to the total and advancing a pointer on the employee list N-1 times.The last salary is added separately. What is the reason for this? Why don't we perform the loop N times? 
\end{problem}

\begin{solution}
  Because if we perform the loop N times, when the machine performs the N$^{th}$ time, first we add the N$^{th}$ employee's salary to the total, and then we point to the next of the N$^{th}$ employee, namely the (N+1)$^{th}$ employee.\\
  So here comes the problem.\\
  The pointer excceeds the limitation of the number of the employees.
\end{solution}
%%%%%%%%%%

%%%%%%%%%%
\begin{problem}[DH:2.2]
  Consider the bubblesort algorithm presented in the text.\\
  (a) Explain why the outer loop is performed only N-1 times.\\
  (b) Improve the algorithm so that on every repeated execution of the outer loop,the inner loop checks one element less.\\
\end{problem}

\begin{solution}
  (a) After each outer loop,it places the largest number of the remaining unordered numbers at the beginning of the ordered numbers.\\
      So when the last N-1 numbers are settled, the second number must be larger than the first number.\\
      So there is no need to perform the N$^{th}$ loop to settle the first number.\\
  (b) \\
\begin{verbatim}
      (1)set i=0;
      (2)do the following N-1 times:
          (2.1) point to the first element;
          (2.2) do the following (N-1-i) times:
                (2.2.1) compare the element pointed to with the next element;
                (2.2.2) if the compared elements are in the wrong order, exchange them;
                (2.2.3) point to the next element;
          (2.3) let i++;
\end{verbatim}
\end{solution}
%%%%%%%%%%
% \newpage  % continue in a new page
%%%%%%%%%%
\begin{problem}[DH:2.3]
Prepare flowcharts for the bubblesort algorithm presented in the text and for the improved version you were asked to design in Exercise 2.2.\\
\end{problem}

% \begin{remark}	
%   Refer to book.
% \end{remark}

\begin{solution}
    \fig{width = 0.80\textwidth}{bubblesort.png}{bubblesort}

    \fig{width = 0.80\textwidth}{bubblesort improved.png}{bubblesort improved}
\end{solution}
%%%%%%%%%%


\begin{problem}[DH:2.4]
Write algorithms that ,given an integer N and a list L of N integers, produce in S and P the sum of the even numbers appearing in L and the product of the odd ones,respectively.\\
(a) Using bounded iteration.\\
(b) Using "goto" statements.
\end{problem}

\begin{solution}
\begin{verbatim}
(a)
(1)set S=0;P=1;
(2) point to the first element;
(3) do the following N-1 times:
(3.1) if if the number is even then add the number to S,else mutiply the number to P;
(3.2) point to the next element;
(4) if the number is even then add the number to S,else mutiply the number to P;
(5) Print S and P;

(b)
(1)set S=0;P=1;
(2) point to the first element;
(3) do the following N-1 times:
(3.1) if if the number is even then goto (3.2),else goto(3.3);
(3.2) add the number to S; goto(3.4)
(3.3) mutiply the number to P;
(3.4) point to the next element;
(4) if the number is even then add the number to S,else mutiply the number to P;
(5) Print S and P;
\end{verbatim}
\end{solution}



\begin{problem}[DH:2.5] % NOTE: use '[]' (instead of '()' or '{}') to provide additional information
Show how to perform the following simulations of some control constructs by others. The sequencing construct “and-then” is implicitly available for all the simulations. You may introduce and use new variables and labels if necessary.\\
(a) Simulate a “for-do” loop by a “while-do” loop.\\
(b) Simulate the “if-then” and “if-then-else” statements by “while-do” loops.\\
(c) Simulate a “while-do” loop by “if-then” and “goto” statements.\\
(d) Simulate a “while-do” loop by a “repeat-until” loop and “if-then” statements.
\end{problem}

\begin{solution}
\begin{verbatim}
(a) for (i=0;i<=N;i++)  do A;


    i=0;
    while (i<=N)
    {
      do A;
      i++;
    }

(b) if (A) then 
        do B;
    else 
        do C;


    while(A)
    {
        do B;
    }
    while{not A}
    {
        do C;
    }

(c) while(A)
    {
        do B;
    }


    (1)if (A) then goto(2) else goto(3);
    (2) do B; goto (1);
    (3) done! exit;

(d) while(A)
    {
      do B;
    }


    if A then
    {
      repeat
      {
        do B
      }until(not A)
    }
\end{solution}
\end{verbatim}
\end{solution}
%%%%%%%%%%

%%%%%%%%%%
\begin{problem}[DH: 2.6]
Write down the sequence of moves resolving the Towers of Hanoi problem for five rings.\\
\end{problem}

\begin{solution}
\begin{lstlisting}
//
//  main.cpp
//  Towers of Hanoi
//
//  Created by 丁保荣 on 2017/10/21.
//  Copyright © 2017年 丁保荣. All rights reserved.
//

#include <iostream>
using namespace std;
void move(int start,int num, char from,char to,char via);
int main(void)
{
    int n;
    cin >> n;
    move(1,n,'A','B','C');
    return 0;
}

void move(int start, int num, char from, char  to, char via)
{
    if(num==1)
    {
        cout <<"move "<< from<<" to "<< to<<endl;
        return;
    }
    else
    {
        move(start,num-1,from,via,to);
        cout<<"move "<< from<<" to "<<to <<endl;
        move(start,num-1,via,to,from);
    }
}
\end{lstlisting}
\begin{verbatim}
move A to B
move A to C
move B to C
move A to B
move C to A
move C to B
move A to B
move A to C
move B to C
move B to A
move C to A
move B to C
move A to B
move A to C
move B to C
move A to B
move C to A
move C to B
move A to B
move C to A
move B to C
move B to A
move C to A
move C to B
move A to B
move A to C
move B to C
move A to B
move C to A
move C to B
move A to B
\end{verbatim}
\end{solution}
%%%%%%%%%%
% \newpage  % continue in a new page
%%%%%%%%%%
\begin{problem}[DH: 2.7]
The factorial of a non-negative integer N is the product of all positive integers smaller than or equal to N. More formally, the expression N factorial, denoted by N!, is recursively defined by \(0! = 1\) and \((N + 1)! = N! × (N + 1)\). For example, \(1! = 1\) and \(4! = 3! × 4 = . . . = 1 × 2 × 3 × 4 = 24\).\\
\\
Write algorithms that compute \(N!\), given a non-negative integer \(N\).\\
\\
(a) Using iteration statements.\\
(b) Using recursion.
\end{problem}

% \begin{remark}  
%   Refer to book.
% \end{remark}

\begin{solution}
\begin{verbatim}
(a)
(1)set fac=1;
(2)if (N>1) then
      {
        (2.1)do the following N-1 times:
            (2.1.1) fac=fac*N;
            (2.1.1) N--;

      } 
(3)print fac;

(b)
fac(n):
(1) if (n=1 or n=0) then fac(n)=1
    else fac(n)=fac(n-1)*n;

main():
(1) fac(N);
\end{verbatim}
\end{solution}

\begin{problem}[DH: 2.8]
Show how to simulate a “while-do” loop by conditional statements and a recursive procedure.\\
\end{problem}

\begin{solution}
\begin{verbatim}
while(A) do B;


rec():
(1) if (A) then
    {
      do B;
      rec();
    }
main():
(1) rec();



\end{verbatim}


\end{solution}
%%%%%%%%%%%%%%%%%%%%
\begincorrection	% begin the ``correction'' part (Optional)

%%%%%%%%%%
\begin{problem}[UD:4.5]
  Negate the following sentences. If you don’t know how to negate it,change it to symbols and then negate. State the universe, if appropriate.\\
  (j) For all $\varepsilon$>0,there exist $\delta$>0 such that if x is a real number with |x−1|<$\delta$,then |$x^2$ −1| < $\varepsilon$.\\
  (k) For all real numbers M,there exists a real number N such that |f(n)|>M for all n > N.
\end{problem}


% Or use the ``solution'' environment
\begin{revision}
   (j) There exists a $\varepsilon$ >0 ,for every $\delta$ >0 ,such that if x is a real number with |x-1|<$\delta$,then |$x^2$ -1| $\geq \varepsilon$.\\
   (k) 批改有误
\end{revision}
%%%%%%%%%%

\begin{problem}[UD:4.7]
Consider the following statement:\\
$\forall x,((x  \in  Z \wedge \neg (\exists y,(y \in Z \wedge x = 7y))) \rightarrow (\exists z,(z \in Z \wedge x = 2z))).$\\
(a) Negate this statement.\\
\end{problem}

\begin{revision}
(a) 批改有误
\end{revision}







%%%%%%%%%%%%%%%%%%%%
\beginfb	% begin the feedback section (Optional)

你可以写:
\begin{itemize}
  \item 对课程及教师的建议与意见
  \item 教材中不理解的内容
  \item 希望深入了解的内容
  \item 等
\end{itemize}
%%%%%%%%%%%%%%%%%%%%
\end{document}