%%%%%%%%%%%%%%%%%%%%%%%%%%%%%%%%%%%%%%%%%%%%%%%%%%%%%%%%%%%%
% File: hw.tex 						   %
% Description: LaTeX template for homework.                %
%
% Feel free to modify it (mainly the 'preamble' file).     %
% Contact hfwei@nju.edu.cn (Hengfeng Wei) for suggestions. %
%%%%%%%%%%%%%%%%%%%%%%%%%%%%%%%%%%%%%%%%%%%%%%%%%%%%%%%%%%%%

%%%%%%%%%%%%%%%%%%%%%%%%%%%%%%%%%%%%%%%%%%%%%%%%%%%%%%%%%%%%%%%%%%%%%%
% IMPORTANT NOTE: Compile this file using 'XeLaTeX' (not 'PDFLaTeX') %
%
% If you are using TeXLive 2016 on windows,                          %
% you may need to check the following post:                          %
% https://tex.stackexchange.com/q/325278/23098                       %
%%%%%%%%%%%%%%%%%%%%%%%%%%%%%%%%%%%%%%%%%%%%%%%%%%%%%%%%%%%%%%%%%%%%%%

\documentclass[11pt, a4paper, UTF8]{ctexart}
\input{preamble}  % modify this file if necessary

%%%%%%%%%%%%%%%%%%%%
\title{第十三讲:布尔代数}
\me{丁保荣}{171860509}
\date{\today}     % you can specify a date like ``2017年9月30日''.
%%%%%%%%%%%%%%%%%%%%
\begin{document}
\maketitle
%%%%%%%%%%%%%%%%%%%%
\noplagiarism	% always keep this
%%%%%%%%%%%%%%%%%%%%
\beginthishw	% begin ``this homework (hw)'' part

%%%%%%%%%%

\begin{problem}[第一题]
证明布尔代数是有界有补分配格,有界有补分配格是布尔代数
\end{problem}

\begin{solution}
(1) prove:A Boolean algebra B is a bounded, distributive and complemented lattice.\\
By Theorem 15.2 and axiom [B1], every Boolean algebra B satisfies the associative, commutative, and absorption laws and hence is a lattice where + and ∗ are the join and meet operations, respectively. With respect to this lattice, a+1 = 1 implies a ≤ 1 and a ∗ 0 = 0 implies 0 ≤ a,for any element a$\in$ B. Thus B is a bounded lattice. Furthermore, axioms [B2] and [B4] show that B is also distributive and complemented.\\
(i) Commutive laws: $\because$ a+b=b+a and a*b=b*a, $\therefore$ a$\vee$b=b$\vee$a and a$\wedge$b=b$\wedge$a\\
(ii) Associative laws: $\because$ (a+b)+c=a+(b+c) and (a*b)*c=a*(b*c), $\therefore$ (a$\vee$b)$\vee$c=a$\vee$(b$\vee$c) and (a$\wedge$b)$\wedge$c=a$\wedge$(b$\wedge$c)\\
(iii) Absorption laws: $\because$ a*(a+b)=a and a+(a*b)=a, $\therefore$ a$\wedge$(a$\vee$b)=a and a$\vee$(a$\wedge$b)=a\\
(iv) bounded: a+1 = 1 implies a ≤ 1 and a ∗ 0 = 0 implies 0 ≤ a,for any element a$\in$ B. Thus B is a bounded lattice.\\
(v) According to [B2], we can conclude that a$\vee$(b$\wedge$c)=(a$\vee$b)$\wedge$(a$\vee$c) and a$\wedge$(b$\vee$c)=(a$\wedge$b)$\vee$(a$\wedge$c)\\
(vi) According to [B4], we can conclude that a$\vee$a'=1 and a$\wedge$a'=0\\
$\therefore$ A Boolean algebra B is a bounded, distributive and complemented lattice.\\
(2) prove:A bounded, distributive and complemented lattice B ia s Boolean algebra.\\
a lattice where + and ∗ are the join and meet operations\\
(i) communitive laws: $\because$ a$\vee$b=b$\vee$a and a$\wedge$b=b$\wedge$a, $\therefore$ a+b=b+a and a*b=b*a\\
(ii) Distributive laws: $\because$ a$\vee$(b$\wedge$c)=(a$\vee$b)$\wedge$(a$\vee$c) and a$\wedge$(b$\vee$c)=(a$\wedge$b)$\vee$(a$\wedge$c),$\therefore$ a+(b∗c)=(a+b)∗(a+c) and a∗(b+c)=(a ∗ b)+(a ∗ c)\\
(iii) Identity laws: $\because$ 0 is the infimun of B and 1 is the supremum of B, $\therefore$ a$\vee$0=a and a$\wedge$1=a, $\therefore$ a+0=a and a*1=a\\
(iv) Complement laws: $\because$ a' is the complement of a, $\therefore$ a$\vee$a'=1 and a$\wedge$a'=0,$\therefore$ a+a'=1 and a*a'=0\\
$\therefore$ A bounded, distributive and complemented lattice B ia s Boolean algebra.\\
Based on (1) and (2),布尔代数是有界有补分配格,有界有补分配格是布尔代数.
\end{solution}

\begin{problem}[第二题]
证明SM定理15.6
\end{problem}
\begin{solution}
a=u$_1$+u$_2$+$\cdots$+u$_r$ and b=v$_1$+v$_2$+$\cdots$+v$_s$ and a'=w$_1$+w$_2$+$\cdots$+w$_t$ and \{u$_1$,u$_2$,$dots$,u$_r$\}$\cap$\{w$_1$,w$_2$,$dots$,w$_t$\}=$\emptyset$ and \{u$_1$,u$_2$,$dots$,u$_r$\}$\cup$\{w$_1$,w$_2$,$dots$,w$_t$\}= A \\
and a+b=u$_1$+u$_2$+$\cdots$+u$_r$+v$_1$+v$_2$+$\cdots$+v$_s$\\
and a*b=x$_1$+x$_2$+$\cdots$+x$_q$(x$_1$,x$_2$,$\dots$,x$_q$ all are the common elements of \{u$_1$,u$_2$,$dots$,u$_r$\} and \{v$_1$,v$_2$,$dots$,v$_s$\})\\
$\therefore$ f(a)=\{u$_1$,u$_2$,$dots$,u$_r$\} and f(b)=\{v$_1$,v$_2$,$dots$,v$_s$\} and f(a')=\{w$_1$,w$_2$,$dots$,w$_t$\}\\
and f(a+b)=\{u$_1$,u$_2$,$dots$,u$_r$\}$\cup$\{v$_1$,v$_2$,$dots$,v$_s$\} and f(a*b)=\{u$_1$,u$_2$,$dots$,u$_r$\}$\cap$\{v$_1$,v$_2$,$dots$,v$_s$\}\\
$\therefore$ f(a+b)=f(a)+f(b), and f(a*b)=f(a)*f(b) and f(a')=f(a)’\\
$\therefore$ The above mapping f:B$\rightarrow$P(A) is an isomorphism.
\end{solution}

\begin{problem}[第三题]
证明等势的布尔代数均同构
\end{problem}
\begin{solution}
We define Boolean Algebra A and B have the same number of elements.\\
$\therefore$ according to the problem solved above, we can note that A is isomorphic to P(C) with the mapping f where C is the set of atoms of A and B is isomorphic to P(D) with the mapping g where D is the set of atoms of B.\\
And according to Corollary 15.7, we note that C and D have the same number of elements. \\
$\therefore$ As proved before, we can define a bijective function h: S$\rightarrow$T is a isomorphism.\\
$\therefore$ g$^{-1}$$\circ$h$\circ$f if is a isomorphism.\\
$\therefore$ A and B are isomorphic.\\

\end{solution}




%%%%%%%%%%
%%%%%%%%%%%%%%%%%%%%
\begincorrection	% begin the ``correction'' part (Optional)

%%%%%%%%%%
\begin{problem}[题号]
  题目。
\end{problem}

\begin{cause}
  简述错误原因(可选)。
\end{cause}

% Or use the ``solution'' environment
\begin{revision}
  正确解答。
\end{revision}
%%%%%%%%%%
%%%%%%%%%%%%%%%%%%%%
\beginfb	% begin the feedback section (Optional)

你可以写:
\begin{itemize}
\item SM第十四章中,书上关于最大(maximal)最小(minimal)元的定义与ppt中关于他们的定义不一致,极大极小直的定义也产生了冲突,以哪一个为准?\\
\item 想问一下,我们以后的课程(包括除问题求解外的课程)会学《深入理解计算机系统》这本书吗?
\end{itemize}
%%%%%%%%%%%%%%%%%%%%
\end{document}