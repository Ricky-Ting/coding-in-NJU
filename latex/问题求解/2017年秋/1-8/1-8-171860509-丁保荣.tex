%%%%%%%%%%%%%%%%%%%%%%%%%%%%%%%%%%%%%%%%%%%%%%%%%%%%%%%%%%%%
% File: hw.tex 						   %
% Description: LaTeX template for homework.                %
%
% Feel free to modify it (mainly the 'preamble' file).     %
% Contact hfwei@nju.edu.cn (Hengfeng Wei) for suggestions. %
%%%%%%%%%%%%%%%%%%%%%%%%%%%%%%%%%%%%%%%%%%%%%%%%%%%%%%%%%%%%

%%%%%%%%%%%%%%%%%%%%%%%%%%%%%%%%%%%%%%%%%%%%%%%%%%%%%%%%%%%%%%%%%%%%%%
% IMPORTANT NOTE: Compile this file using 'XeLaTeX' (not 'PDFLaTeX') %
%
% If you are using TeXLive 2016 on windows,                          %
% you may need to check the following post:                          %
% https://tex.stackexchange.com/q/325278/23098                       %
%%%%%%%%%%%%%%%%%%%%%%%%%%%%%%%%%%%%%%%%%%%%%%%%%%%%%%%%%%%%%%%%%%%%%%

\documentclass[11pt, a4paper, UTF8]{ctexart}
\input{preamble}  % modify this file if necessary

%%%%%%%%%%%%%%%%%%%%
\title{第八讲:集合及其运算}
\me{丁保荣}{171860509}
\date{\today}     % you can specify a date like ``2017年9月30日''.
%%%%%%%%%%%%%%%%%%%%
\begin{document}
\maketitle
%%%%%%%%%%%%%%%%%%%%
\noplagiarism	% always keep this
%%%%%%%%%%%%%%%%%%%%
\beginthishw	% begin ``this homework (hw)'' part

%%%%%%%%%%
\begin{problem}[UD: 6.7]
Find an expression for each of the shaded sets in the Venn diagrams of Figure 6.5
\end{problem}

\begin{solution}
from left to right, from up to down\\
(a) B $\backslash$ (A $\cap$ B)\\
(b) (A $\cup$ B) $\backslash$ (A $\cap$ B)\\
(c) A $\cap$ B $\cap$ C\\
(d) (B $\cap$ C) $\backslash$ A\\
(e)  ((A$\cap$B)$\cup$(A$\cap$C)$\cup$(B$\cap$C)) $\backslash$ (A $\cap$ B $\cap$ C)
\end{solution}


\begin{problem}[UD: 6.16]
In each part of this problem, two sets, A and B, are defined. Prove that A$\Subset$B in each of the following:\\
(a) A=\{x$^2$ :x$\in$ Z\} and B=Z;\\
(b) A=R and B=\{2x : x$\in$ R\};\\
(c) A=\{(x,y)$\in$ R$^2$ : y=(5-3x)/2\} and B=\{(x,y)$\in$ R$^2$ : 2y+3x=5\}.\\
\end{problem}

\begin{solution}
(a)\\
$\because$ x$\in$ Z\\
$\therefore$ x=2k or 2k+1 (k$\in$ Z)\\
(1) if x=2k, then x$^2$=4k$^2$ , $\therefore$ 4k$^2$ $\in$ Z, $\therefore$ x $\in$ Z\\
(2) if x=2k+1 then x$^2$=4k$^2$+4k+1, $\therefore$ 4k$^2$+4k+1 $\in$ Z, $\therefore$ x$\in$ Z\\
Based on (1) and (2) we can conclude that for all x $\in$ Z ,we have x$^2$ $\in$ Z, namely B\\
$\therefore$ A$\Subset$ B\\
\\
(b)\\
for every y $\in$ R,namely A, we have y/2 $\in$ R\\
we can let y=2x\\
$\therefore$ for every 2x $\in$ R,namely A, we have x $\in$ R\\
$\therefore$ that property satisfies the defination of B\\
$\therefore$ for every 2x $\in$ R,namely A, we have 2x $\in$ B\\
$\therefore$ A$\Subset$ B\\
\\
(c)\\
for every (x,y) $\in$ A, we have y=(5-3x)/2\\
$\therefore$ for every (x,y) $\in$ A, we have 2y+3x=5\\
$\therefore$ for every (x,y) $\in$ A, we have (x,y)$\in$ B\\
$\therefore$ A$\Subset$B
\end{solution}

\begin{problem}[UD: 6.17]
Prove that one set is a proper subset of the other one in each of the following:\\
(a) A=\{(x,y) $\in$ R$^2$ : xy>0 \} and B=\{(x,y) $\in$ R$^2$ :x$^2 + y ^2$ > 0 \};\\
(b) A=$\emptyset$ and B=\{(x,y)$\in$ R$^2$: $x^2 + y^2 \le 0$ \}.
\end{problem}

\begin{solution}
(a)\\
(1) for every (x,y) $\in$ A,we have xy>0\\
$\therefore$ xy$\not=$ 0\\
$\therefore$ $x^2 + y^2>0$\\
$\therefore$ that property satisfies the defination of B\\
$\therefore$ for every (x,y) $\in$ A,we have (x,y) $\in$ B\\
$\therefore$ A$\Subset$B\\
(2) (2,-5)$\in$ B but (2,-5)$\notin$ A\\
$\therefore$ A $\not=$ B\\
Based on (1) and (2), A$\subsetneqq$ B\\
\\
(b)\\
(1) we need to prove for every x, if x$\in$ A,then x$\in$ B\\
Since there are no elements in the empty set, the antecedent is always false. Therefore the implication is always true.\\
$\therefore$ A $\Subset$ B\\
(2) (0,0) $\in$ B, but (0,0) $\notin$ A\\
$\therefore$ A$\not=$B\\
Based on (1) and (2), A$\subsetneqq$B\\
\end{solution}

\begin{problem}[UD:7.1]
In this problem we refer to statements of Theorem 7.4.\\
(a) Prove statement 2. \\
(A $^\complement$)$^\complement$=A.\\
(b) Prove statement 14. \\
A $\cap$(B $\cup$ C) = (A $\cap B$)$\cup$(A $\cap$ C). (Distributive property)\\
(c) Prove statement 16. \\
X $\backslash$(A $\cap$ B)=(X $\backslash$ A)$\cup$ (X $\backslash$ B).(Demorgan's law)\\
\indent (When X is the universe we also write (A $\cap$ B)$^\complement$ =A$^\complement$ $\cup$ B$^\complement$\\
(d) Prove statement 18.\\
A$\Subset$B if and only if (X $\backslash$ B) $\Subset$ (X $\backslash$ A).\\
\indent (When X is the universe we also write A$\Subset$B if and only if B$^\complement$ $\Subset$ A$^\complement$)\\
(e) Prove statement 20.\\
A $\cap$ B=B if and only if B$\Subset$A.
\end{problem}


\begin{solution}
(a)\\
(1) for every x$\in$ (A$^\complement$)$^\complement$ ,we have x$\notin$ (A$^\complement$)\\
$\therefore$ we have x$\in$ A\\
$\therefore$ (A$^\complement$)$^\complement$ $\Subset$ A\\
(2) for every x$\in$ A,we have x$\notin$ (A$^\complement$)\\
$\therefore$ we have x$\in$ (A$^\complement$)$^\complement$\\
$\therefore$  A $\Subset$ (A$^\complement$)$^\complement$\\
Based on (1) and (2),we have (A $^\complement$)$^\complement$=A.\\
\\
\\
(b)\\
(1) for every x $\in$ A $\cap$(B $\cup$ C),x$\in$ A and x$\in$ (B $\cup$ C)\\
$\therefore$ there are two cases:\\
\indent (i) x$\in$ A and x$\in$ B :\\
\indent \indent $\therefore$ x$\in$(A $\cap$ B)\\
\indent \indent $\therefore$ x$\in$(A $\cap B$)$\cup$(A $\cap$ C)\\
\indent (ii) x$\in$ A and x$\in$ C\\
\indent \indent $\therefore$ x$\in$(A $\cap$ C)\\
\indent \indent $\therefore$ x$\in$(A $\cap B$)$\cup$(A $\cap$ C)\\
according to (i)and(ii),we can conclude that A $\cap$(B $\cup$ C) $\Subset$ (A $\cap B$)$\cup$(A $\cap$ C)\\
(2) for every x $\in$ (A $\cap B$)$\cup$(A $\cap$ C), x$\in$ (A$\cap$B) or x$\in$ (A$\cap$C)\\
$\therefore$ there are two cases:\\
\indent (i) x$\in$ (A$\cap$B)\\
\indent \indent x$\in$ A and x$\in$ B\\
\indent \indent x$\in$ A and x$\in$ (B $\cup$ C)\\
\indent \indent x $\in$ A $\cap$(B $\cup$ C)\\
\indent (ii) x$\in$ (A$\cap$C)\\
\indent \indent x$\in$ A and x$\in$ C\\
\indent \indent x$\in$ A and x$\in$ (B $\cup$ C)\\
\indent \indent x $\in$ A $\cap$(B $\cup$ C)\\
according to (i) and (ii),we can conclude that (A $\cap B$)$\cup$(A $\cap$ C) $\Subset$ A $\cap$(B $\cup$ C)\\
Based on (1) and (2) ,we can conclude that A $\cap$(B $\cup$ C) = (A $\cap B$)$\cup$(A $\cap$ C)\\
\\
\\
(c)\\
(1) for every x $\in$ X $\backslash$(A $\cap$ B),we have x$\in$ X and x$\notin$ (A $\cap$ B)\\
$\therefore$ x $\in$ X and x$\in$ (A$\cap$ B)$^\complement$\\
$\therefore$ x$\in$ X and x$\in$((A$\backslash$(A$\cap$ B)) $\cup$ (B$\backslash$(A$\cap$ B)) $\cup$ (A$\cup$B)$^\complement$)\\
$\therefore$ there are three cases:\\
\indent (i) x$\in$X and x$\in$ (A$\backslash$(A$\cap$ B))\\
\indent \indent $\therefore$ x$\in$ X and x$\notin$ B\\
\indent \indent $\therefore$ x$\in$ (X$\backslash$ B)\\
\indent \indent $\therefore$ x$\in$ (X $\backslash$ A)$\cup$ (X $\backslash$ B)\\
\indent (ii) x$\in$X and x$\in$ (B$\backslash$(A$\cap$ B))\\
\indent \indent $\therefore$ x$\in$ X and x$\notin$ A\\
\indent \indent $\therefore$ x$\in$ (X$\backslash$ A)\\
\indent \indent $\therefore$ x$\in$ (X $\backslash$ A)$\cup$ (X $\backslash$ B)\\
\indent (iii) x$\in$ X and x$\in$ (A$\cup$B)$^\complement$\\
\indent \indent $\therefore$ x$\in$ X and x$\notin$ A and x$\notin$ B\\
\indent \indent $\therefore$ x$\in$ (X $\backslash$ A)$\cap$ (X $\backslash$ B)\\
\indent \indent $\therefore$ x$\in$ (X $\backslash$ A)$\cup$ (X $\backslash$ B)\\
according to (i), (ii) and (iii), we can conclude that X $\backslash$(A $\cap$ B) $\Subset$ (X $\backslash$ A)$\cup$ (X $\backslash$ B)\\
(2)for every x $\in$ (X $\backslash$ A)$\cup$ (X $\backslash$ B),we have x$\in$ X and x$\in$((A$\backslash$(A$\cap$ B)) $\cup$ (B$\backslash$(A$\cap$ B)) $\cup$ (A$\cup$B)$^\complement$)\\
$\therefore$ there are three cases:\\
\indent (i) x$\in$X and x$\in$ (A$\backslash$(A$\cap$ B))\\
\indent \indent $\therefore$ x$\in$ X and x$\in$((A$\backslash$(A$\cap$ B)) $\cup$ (B$\backslash$(A$\cap$ B)) $\cup$ (A$\cup$B)$^\complement$)\\
\indent \indent $\therefore$ x $\in$ X and x$\in$ (A$\cap$ B)$^\complement$\\
\indent \indent $\therefore$ x $\in$ X and x$\notin$ (A$\cap$ B)\\
\indent \indent $\therefore$ x $\in$ X $\backslash$(A $\cap$ B)\\
\indent (ii) x$\in$X and x$\in$ (B$\backslash$(A$\cap$ B))\\
\indent \indent $\therefore$ x$\in$ X and x$\in$((A$\backslash$(A$\cap$ B)) $\cup$ (B$\backslash$(A$\cap$ B)) $\cup$ (A$\cup$B)$^\complement$)\\
\indent \indent $\therefore$ x $\in$ X and x$\in$ (A$\cap$ B)$^\complement$\\
\indent \indent $\therefore$ x $\in$ X and x$\notin$ (A$\cap$ B)\\
\indent \indent $\therefore$ x $\in$ X $\backslash$(A $\cap$ B)\\
\indent (iii) x$\in$ X and x$\in$ (A$\cup$B)$^\complement$\\
\indent \indent $\therefore$ x$\in$ X and x$\in$((A$\backslash$(A$\cap$ B)) $\cup$ (B$\backslash$(A$\cap$ B)) $\cup$ (A$\cup$B)$^\complement$)\\
\indent \indent $\therefore$ x $\in$ X and x$\in$ (A$\cap$ B)$^\complement$\\
\indent \indent $\therefore$ x $\in$ X and x$\notin$ (A$\cap$ B)\\
\indent \indent $\therefore$ x $\in$ X $\backslash$(A $\cap$ B)\\
according to (i), (ii) and (iii), we can conclude that (X $\backslash$ A)$\cup$ (X $\backslash$ B) X $\backslash$(A $\cap$ B) $\Subset$ (X $\backslash$(A $\cap$ B))\\
according to (1) and (2),we can conclude that X $\backslash$(A $\cap$ B)=(X $\backslash$ A)$\cup$ (X $\backslash$ B).\\
\\
\\
(d)\\
(1) Because A$\Subset$ B, we have B$^\complement$ $\Subset$ A$^\complement$\\
for every x$\in$(X$\backslash$B), we have x$\in$ (X $\cap$ B$^\complement$)\\
and $\because$ B$^\complement$ $\Subset$ A$^\complement$\\
$\therefore$ x$\in$ (X $\cap$ A$^\complement$)\\
$\therefore$ (X $\backslash$ B) $\Subset$ (X $\backslash$ A)\\
(2) $\because$ (X $\backslash$ B) $\Subset$ (X $\backslash$ A)\\
$\therefore$ (X $\cap$ B$^\complement$) $\Subset$ (X $\cap$ A$^\complement$)\\
$\therefore$ B$^\complement$ $\Subset$ A$^\complement$\\
$\therefore$ A$\Subset$ B\\
Based on (1) and (2),we can conclude that A$\Subset$B if and only if (X $\backslash$ B) $\Subset$ \\
\\
\\
(e)\\
(1) $\because$ A $\cap$ B=B \\
$\therefore$ for every x$\in$ B, we have x$\in$(A $\cap$ B)\\
$\therefore$ x$\in$ A\\
$\therefore$ B$\Subset$ A\\
(2) $\because$ B$\Subset$ A\\
$\therefore$ for every x $\in$ B, we have x $\in$ A\\
\indent (i)\\
$\therefore$ for x $\in$ (A $\cap$ B), we have x $\in$ B\\
$\therefore$ A$\cap$B $\Subset$ B\\
\indent (ii)\\
$\therefore$ for x $\in$ B,we have x$\in$ A\\
$\therefore$ x$\in$ A$\cap$B\\
$\therefore$ B $\Subset$ (A$\cap$B)\\
accoding to (i) and (ii) ,we can conclude that A$\cap$B=B\\
Based on (1) and (2),we can conclude that A $\cap$ B=B if and only if B$\Subset$A\\

\end{solution}


\begin{problem}[UD:7.8]
Consider the following sets:\\
(i) (A$\cap$B) $\backslash$ (A$\cap$B$\cap$C),\\
(ii) A$\cap$B $\backslash$ (A$\cap$B$\cap$C),\\
(iii) A$\cap$ B $\cap$ C$^\complement$,\\
(iv) (A$\cap$B) $\backslash$ C,and\\
(v) (A$\backslash$C) $\cap$ (B $\backslash$C).\\
(a) Which of the sets above are written ambiguously,if any?\\
(b) Of the sets above that make sense, which ones equal the set sketched in Figure 7.2?\\
(c) Prove that (A$\cap$B)$\backslash$C=(A$\backslash$C) $\cap$ (B$\backslash$C).\\
\end{problem}

\begin{solution}
(a) the second one is written ambiguously.\\
(b) all above except the second one equal the set sketched in Figure 7.2\\
(c)\\
(1) for every x $\in$ (A$\cap$B)$\backslash$C,we have x$\in$ (A $\cap$ B) and x$\notin$ C\\
$\therefore$ x$\in$ (A$\cap$ B) and x$\in$ C$^\complement$\\
$\therefore$ x$\in$ (A $\cap$ B $\cap$ C$^\complement$)\\
$\therefore$ x$\in$ (A$\cap$C$^\complement$) $\cap$ (B$\cap$C$^\complement$)\\
$\therefore$ x$\in$ (A$\backslash$C) $\cap$ (B$\backslash$C)\\
$\therefore$ (A$\cap$B)$\backslash$C $\Subset$ (A$\backslash$C) $\cap$ (B$\backslash$C)\\
(2) for every x $\in$ (A$\backslash$C) $\cap$ (B$\backslash$C),we have x$\in$ (A$\cap$C$^\complement$) $\cap$ (B$\cap$C$^\complement$)\\
$\therefore$ x$\in$ (A $\cap$ B $\cap$ C$^\complement$)\\
$\therefore$ x$\in$ (A$\cap$ B) and x$\in$ C$^\complement$\\
$\therefore$ x$\in$ (A $\cap$ B) and x$\notin$ C\\
$\therefore$ x$\in$ (A$\cap$B)$\backslash$C\\
$\therefore$ (A$\backslash$C) $\cap$ (B$\backslash$C) $\Subset$ (A$\cap$B)$\backslash$C\\
According to (1) and (2),we can conclude that (A$\cap$B)$\backslash$C=(A$\backslash$C) $\cap$ (B$\backslash$C).\\
\end{solution}


\begin{problem}[UD:7.9]
In this problem you will prove that the union of two sets can be rewritten as the union of two disjoint sets.\\
(a) Prove that the two sets A$\backslash$B and B are disjoint.\\
(b) Prove that A$\cup$=(A$\backslash$B) $\cup$ B.
\end{problem}

\begin{solution}
(a)\\
(1) for every x $\in$ A$\backslash$B,we have x$\in$ A and x$\notin$ B\\
$\therefore$ x$\notin$B\\
(2) for every x $\in$ B, we have x$\notin$ B$^\complement$ \\
$\therefore$ x$\notin$ A$\backslash$B\\
Based on (1) and (2),we can conclude that A$\backslash$B and B are disjoint.\\
\\
(b)\\
(1) for every x $\in$ A$\cup$B, we have x$\in$A or x$\in$B\\
for x$\in$ B ,we can show that x$\in$ (A $\backslash$ B) $\cup$ B\\
for x$\in$ A ,there are two cases:\\
\indent (i) x$\in$ A$\cap$ B\\
\indent \indent $\therefore$ x$\in$ B\\
\indent \indent $\therefore$ x$\in$ (A $\backslash$ B) $\cup$ B\\
\indent (ii) x$\in$ A$\backslash$ (A$\cap$B)\\
\indent \indent $\therefore$ x$\in$ A$\backslash$ B\\
\indent \indent $\therefore$ x$\in$ (A $\backslash$ B) $\cup$ B\\
$\therefore$ we can conclude that A$\cup$B $\Subset$ (A$\backslash$B)$\cup$B.\\
(2) for every x $\in$ (A$\backslash$B)$\cup$B,we have x$\in$ (A$\backslash$) or x$\in$ B.\\
\indent (i). x$\in$ (A$\backslash$)\\
\indent \indent $\therefore$ x$\in$ A\\
\indent \indent $\therefore$ x$\in$ A$\cup$B\\
\indent (ii).x$\in$ B\\
\indent \indent $\therefore$ x$\in$ A$\cup$B\\
$\therefore$ we can conclude that (A$\backslash$B)$\cup$B $\Subset$ A$\cup$B\\
Based on (1) and (2), A$\cup$=(A$\backslash$B) $\cup$ B\\
\end{solution}


\begin{problem}[UD:7.10]
Prove or disprove: If A$\cup$B=A$\cup$C, then B=C;
\end{problem}

\begin{solution}
let A=\{1,2,3\} B=\{1\} C=\{2\}\\
then we have A$\cup$B =A$\cup$C =\{1,2,3\},but B$\not=$C\\
So the statement is not true.
\end{solution}


\begin{problem}[UD:7.11]
Prove or give a counterexample for the following statement.\\
\indent Let X be the universe and A,B$\Subset$X. If A$\cap$Y=B$\cap$Y for all Y$\Subset$ X,then A=B.\\
\end{problem}

\begin{solution}
(1) Let Y=A, then we have A$\cap$A=B$\cap$A\\
$\therefore$ A=B$\cap$A\\
$\therefore$ A$\Subset$B\\
(2) Let Y=B, then we have A$\cap$B=B$\cap$B\\
$\therefore$ A$\cap$B=B\\
$\therefore$ B$\Subset$A\\
Based on (1) and (2),we can conclude that A=B\\
So the statement is true.
\end{solution}


\begin{problem}[UD:8.1]
Consider the intervals of real numbers given by A$_n$=[0,1/n),B$_n$=[0,1/n],and C$_n$=(0,1/n).\\
(a)Find $\bigcup_{n=1}^{\infty}$A$_n$ ,$\bigcup_{n=1}^{\infty}$B$_n$,and $\bigcup_{n=1}^{\infty}$C$_n$.\\
(b)Find $\bigcap_{n=1}^{\infty}$A$_n$ ,$\bigcap_{n=1}^{\infty}$B$_n$,and $\bigcap_{n=1}^{\infty}$C$_n$.\\
(c)Does $\bigcup_{n\in N}$A$_n$ make sense? Why or why not?
\end{problem}
\begin{solution}
(a)\\
$\bigcup_{n=1}^{\infty}$A$_n$=[0,1)\\
$\bigcup_{n=1}^{\infty}$B$_n$=[0,1]\\
$\bigcup_{n=1}^{\infty}$C$_n$=(0,1)\\
\\
(b)\\
$\bigcap_{n=1}^{\infty}$A$_n$=\{0\}\\
$\bigcap_{n=1}^{\infty}$B$_n$=\{0\}\\
$\bigcap_{n=1}^{\infty}$C$_n$=$\emptyset$\\
\\
(c)\\
It doesn't make sense.\\
$\because$ 0$\in$N, but 0 cannot be the denominator.\\

\end{solution}






\begin{problem}[UD:8.4]
Prove or give a counterexample:Let \{A$_n$:n$\in$Z$^+$\} and \{B$_n$:n$\in$Z$^+$\} be two indexed families of set. If A$_n$ $\subsetneq$ B$_n$ for all n$\in$ Z$^+$, then \\
\indent \indent $\bigcap_{n=1}^{\infty}$A$_n$ $\subsetneq$ $\bigcap_{n=1}^{\infty}$B$_n$.\\
(Recall that A$\subsetneq$B means strict inclusion; that is, A$\Subset$B and A$\not=$B.)
\end{problem}

\begin{solution}
This statement is not true.\\
We can let B$_n$=\{1,2\}\\
when n is odd let A$_n$=\{1\}\\
when n is even let A$_n$=\{2\}\\
$\therefore$ for all n$\in$ Z$^+$, we have A$_n$ $\subsetneq$ B$_n$\\
but $\bigcap_{n=1}^{\infty}$A$_n$ = $\bigcap_{n=1}^{\infty}$B$_n$
$\therefore$ the statement is not true.\\
\end{solution}










\begin{problem}[UD:8.7]
Suppose that \{A$_\alpha$:$\alpha$$\in$I\} is an indexed family of subsets of a set X, and that B is a subset of X.\\
(a) If A$_\alpha$ =$\emptyset$ for some $\alpha$ $\in$ I,prove that $\bigcap_{\alpha \in I}$A$_\alpha$ =$\emptyset$.\\
(b) If A$_\alpha$ =X for some $\alpha$ $\in$ I,prove that $\bigcup_{\alpha \in I}$A$_\alpha$ =X.\\
(c) If B$\Subset$A$_\alpha$ for every $\alpha$ $\in$ I, prove that B$\Subset$ $\bigcap_{\alpha \in I}$A$_\alpha$.\\
\end{problem}

\begin{solution}
(a) Suppose that $\bigcap_{\alpha \in I}$A$_\alpha$ $\not=$ $\emptyset$, we have there exists an x$\in$ $\bigcap_{\alpha \in I}$A$_\alpha$ =$\emptyset$\\
$\therefore$ we have there exists an x $\in$ A$_\alpha$ for all $\alpha$ $\in$ I\\
$\therefore$ A$_\alpha$ $\not=$ $\emptyset$ for all $\alpha$ $\in$ I\\
but it contradicts with the condition that A$_\alpha$ =$\emptyset$ for some $\alpha$ $\in$ I\\
$\therefore$ what we suppose is wrong \\
$\therefore$ $\bigcap_{\alpha \in I}$A$_\alpha$ =$\emptyset$\\
\\
(b) Suppose that $\bigcup_{\alpha \in I}$A$_\alpha$ $\not=$ X, we have there exists an x $\in$ X but $\notin$ $\bigcup_{\alpha \in I}$A$_\alpha$\\
$\therefore$ there exists an x$\in$ X but $\notin$ A$_\alpha$ for all $\alpha$ $\in$ I\\
but it contradicts with the condition that A$_\alpha$ =X for some $\alpha$ $\in$ I\\
$\therefore$ what we suppose is wrong\\
$\therefore$ $\bigcup_{\alpha \in I}$A$_\alpha$ =X\\
\\
(c) Suppose that B$\nsubseteq$ $\bigcap_{\alpha \in I}$A$_\alpha$, we have there exists an x$\in$ B but $\notin$ $\bigcap_{\alpha \in I}$A$_\alpha$.\\
$\therefore$ there exists an x$\in$ B but $\notin$ A$_\alpha$ for some $\alpha$ $\in$ I\\
but it contradicts with the condition that B$\Subset$A$_\alpha$ for every $\alpha$ $\in$ I\\
$\therefore$ what we suppose is wrong\\
$\therefore$ B$\Subset$ $\bigcap_{\alpha \in I}$A$_\alpha$
\end{solution}








\begin{problem}[UD:8.8]
Define\\
\indent \indent A=R$\backslash$ $\bigcap_{n\in Z^+}$ (R$\backslash$ \{-n,-n+1,...,0,...,n-1,n\}).\\
The set A should be familiar to you. Guess what it is and then prove that your guess is correct.
\end{problem}


\begin{solution}
A=Z\\
\\
According to Exercise 8.9, we have $\bigcap_{n \in Z^+}$ (R$\backslash$ \{-n,-n+1,...,0,...,n-1,n\})=R $\backslash$ ($\bigcup_{n\in Z^+}$ \{-n,-n+1,...,0,...,n-1,n\})
$\therefore$ A=R$\backslash$ (R $\backslash$ Z)\\
$\therefore$ A=R$\backslash$ $\complement_{R}^{Z}$\\
$\therefore$ A=Z
\end{solution}



\begin{problem}[UD:8.9]
Guess a simpler way to express the set A defined as\\
\indent \indent A=Q$\backslash$ $\bigcap_{n\in Z}$(R$\backslash$\{2n\}),\\
and then prove that your guess is correct.
\end{problem}



\begin{solution}
A=\{2n|n$\in$Z\}\\
According to Exercise 8.9, we have $\bigcap_{n\in Z}$(R$\backslash$\{2n\})=R$\backslash$ $\bigcup_{n \in Z}$\{2n\}\\
$\therefore$ A= Q$\backslash$ (R$\backslash$ \{2n|n$\in$Z\})\\
$\therefore$ A= Q$\backslash$ $\complement_{R}^{\{2n|n \in Z\}}$\\
$\therefore$ A=\{2n|n$\in$Z\}\\
\end{solution}




\begin{problem}[UD:8.11]
A collection of sets\{A$_\alpha$:$\alpha \in$ I\} is said to be a pairwise disjoint collection if the following is satisfied :For all $\alpha \beta \in$I, if A$_\alpha$ $\cap$A$_\beta$ $\not=$ $\emptyset$, then A$_\alpha$=A$_\beta$. Suppose that each set A$_\alpha$ is nonempty.\\
(a) Give an example of pairwise disjoint sets A$_1$,A$_2$,A$_3$,....\\
(b) What is the contrapositive of "if A$_\alpha$ $\cap$ A$\beta$ $\not=$ $\emptyset$, then A$_\alpha$=A$_\beta$"?\\
(c) What is the converse of "if A$_\alpha$ $\cap$ A$\beta$ $\not=$ $\emptyset$, then A$_\alpha$=A$_\beta$"?\\
(d) If \{A$_\alpha:\alpha \in I$\} is a pairwise disjoint collection, does the assertion you found in (b) hold for all $\alpha$ and $\beta$ in I?\\
(e) If the assertion that you found in (b) holds for all $\alpha$ and $\beta$ in I, is \{A$_\alpha:\alpha \in I$\} a pairwise disjoint collection?\\
(f) If \{A$_\alpha:\alpha \in I$\} is a pairwise disjoint collection of sets, does it follow that $\bigcap_{\alpha \in I}$A$_\alpha$ =$\emptyset$?\\
(g) If $\bigcap_{\alpha \in I}$A$_\alpha$ =$\emptyset$, is \{A$_\alpha:\alpha \in I$\} necessarily a pairwise disjoint collection of sets?
\end{problem}


\begin{solution}
(a) A$_n$=\{1\}\\
(b) if A$_\alpha$ $\not=$ A$_\beta$, then A$_\alpha$ $\cap$ A$\beta$ = $\emptyset$\\
(c) if A$_\alpha$=A$_\beta$, then A$_\alpha$ $\cap$ A$\beta$ $\not=$ $\emptyset$\\
(d) yes\\
(e) yes\\
(f) no\\
(g) no\\

\end{solution}




\begin{problem}[UD:9.2]
(a) Show that P(A)$\cup$P(B)$\Subset$P(A$\cup$B).\\
(b) Show that P(A)$\cup$P(B)$\not=$P(A$\cup$B) by exhibiting two concrete sets, A and B, for which the aforementioned inequality holds.\\
\end{problem}


\begin{solution}
(a)\\
for every x$\in$ P(A)$\cup$P(B), we have x$\Subset$ A or x$\Subset$ B.\\
$\therefore$ x$\Subset$ A$\cup$B\\
$\therefore$ x$\in$ P(A$\cup$B)\\
$\therefore$ P(A)$\cup$P(B)$\Subset$P(A$\cup$B)\\
\\
(b)\\
Let A=\{1\} B=\{2\} A$\cup$B=\{1,2\}\\
P(A)=\{\{1\},$\emptyset$\}, P(B)=\{\{2\},$\emptyset$\}, P(A$\cup$B)=\{\{1\},\{2\},\{1,2\},$\emptyset$\}\\
P(A)$\cup$P(B)=\{\{1\},\{2\},$\emptyset$\}\\
$\therefore$ P(A)$\cup$P(B)$\not=$P(A$\cup$B)\\

\end{solution}




\begin{problem}[UD:9.4]
Show that A$\Subset$B if and only if P(A) $\Subset$ P(B).
\end{problem}


\begin{solution}
(1) for every x$\in$ P(A), we have x$\Subset$ A\\
$\therefore$ x$\Subset$B\\
$\therefore$ x$\in$ P(B)\\
$\therefore$ we can conclude that P(A) $\Subset$ P(B)\\
(2) for every x$\Subset$ A, we have x$\in$ P(A)\\
$\therefore$ x$\in$ P(B)\\
$\therefore$ x$\Subset$B\\
$\therefore$ we can conclude that A$\Subset$B\\
Based on (1) and (2), A$\Subset$B if and only if P(A) $\Subset$ P(B)\\
\end{solution}





\begin{problem}[UD:9.12]
(a) Prove the following:\\
\indent Let A,B,C and D be nonempty sets. Then A $\times$ B=C $\times$ D if and only if A=C and B=D.\\
(b) Where did your proof use the fact that the sets were nonempty?
\end{problem}

\begin{solution}
(a)\\
(1)\\
$\because$ A,B,C and D are nonempty sets\\
$\therefore$ A$\times$B=\{(x,y)|x$\in$A,y$\in$B\} $\not=$ $\emptyset$ \\
\indent C$\times$D=\{(z,w)|z$\in$C,w$\in$D\} $\not=$ $\emptyset$ \\
$\because$ A $\times$ B=C $\times$ D\\
$\therefore$ for every (x,y) $\in$ A$\times$B ,we can find (x,y) $\in$ C$\times$D\\
\indent for every (z,w) $\in$ C$\times$D ,we can find (z,w) $\in$ A$\times$B\\
$\therefore$ for every x $\in$A, we can find x$\in$C (this is also the same for y)\\
\indent for every z$\in$C,we can find z$\in$ A(this is also the same for w)\\
$\therefore$ A$\Subset$C and C$\Subset$ A, B$\Subset$D and D$\Subset$ B\\
$\therefore$ A=C,B=D\\
(2)\\
$\because$ A,B,C and D are nonempty sets\\
$\therefore$ A$\times$B=\{(x,y)|x$\in$A,y$\in$B\} $\not=$ $\emptyset$ \\
\indent C$\times$D=\{(z,w)|z$\in$C,w$\in$D\} $\not=$ $\emptyset$ \\
$\because$ A=C and B=D\\
$\therefore$ A$\Subset$C and C$\Subset$ A, B$\Subset$D and D$\Subset$ B\\
$\therefore$ for every x $\in$A, we can find x$\in$C (this is also the same for y)\\
\indent for every z$\in$C,we can find z$\in$ A(this is also the same for w)\\
$\therefore$ for every (x,y) $\in$ A$\times$B ,we can find (x,y) $\in$ C$\times$D\\
\indent for every (z,w) $\in$ C$\times$D ,we can find (z,w) $\in$ A$\times$B\\
$\therefore$ A $\times$ B$\Subset$C $\times$ D and C $\times$ D $\Subset$  A $\times$ B\\
$\therefore$ A $\times$ B=C $\times$ D\\
Based on (1) and (2),Let A,B,C and D be nonempty sets. Then A $\times$ B=C $\times$ D if and only if A=C and B=D.\\
\\
(2)\\
when I want to illustrate that A$\times$B $\not=$ $\emptyset$ and C$\times$D $\not=$ $\emptyset$\\

\end{solution}




\begin{problem}[UD:9.13]
Suppose A,B,C and D are four sets. If A$\times$B$\Subset$C$\times$D, must A$\Subset$C and B$\Subset$D? Why or why not?
\end{problem}

\begin{solution}
No, it doesn't need to.\\
We can let C$\Subset$ A and A$\not=$C and B=D=$\emptyset$\\
We can still have A$\times$B$\Subset$C$\times$D\\

\end{solution}






\begin{problem}[UD:9.14]
Let A,B and C be sets. If the statements below are true prove them.\\
If they are false,give a counterexample:\\
(a) A$\times$(B$\cup$C) = (A$\times$B) $\cup$ (A$\times$C);\\
(b) A$\times$(B$\cap$C) = (A$\times$B) $\cap$ (A$\times$C);\\
\end{problem}

\begin{solution}
(a)\\
(1)for every (x,y)$\in$ A$\times$(B$\cup$C), we have x$\in$ A and y$\in$ (B$\cup$C)\\
$\therefore$ there are two cases:\\
\indent (i) x$\in$ A and y$\in$B\\
\indent \indent $\therefore$ (x,y)$\in$ A$\times$B\\
\indent \indent $\therefore$ (x,y)$\in$ (A$\times$B)$\cup$(A$\times$C)\\
\indent (ii) x$\in$ A and y$\in$C\\
\indent \indent $\therefore$ (x,y)$\in$ A$\times$C\\
\indent \indent $\therefore$ (x,y)$\in$ (A$\times$B)$\cup$(A$\times$C)\\
$\therefore$ according to (i) and (ii),A$\times$(B$\cup$C) $\Subset$ (A$\times$B) $\cup$ (A$\times$C)\\
(2)for every (x,y) $\in$ (A$\times$B) $\cup$ (A$\times$C), we have (x,y)$\in$ A$\times$B or(x,y)$\in$ A$\times$C\\
$\therefore$ therefore there are two cases:\\
\indent (i)(x,y)$\in$ A$\times$B\\
\indent \indent x$\in$A and y$\in$B\\
\indent \indent x$\in$ A and y$\in$ (B$\cup$C)\\
\indent \indent (x,y)$\in$ A$\times$(B$\cup$C)\\
\indent (ii)(x,y)$\in$ A$\times$C\\
\indent \indent x$\in$A and y$\in$C\\
\indent \indent x$\in$ A and y$\in$ (B$\cup$C)\\
\indent \indent (x,y)$\in$ A$\times$(B$\cup$C)\\
$\therefore$ according to (i) and (ii),(A$\times$B) $\cup$ (A$\times$C) $\Subset$ A$\times$(B$\cup$C)\\
Based on (1) and (2),A$\times$(B$\cup$C) = (A$\times$B) $\cup$ (A$\times$C)\\
\\
(b)\\
(1)for every (x,y)$\in$ A$\times$(B$\cap$C), we have x$\in$A and y$\in$ B$\cap$C\\
$\therefore$ we have (x$\in$A and y$\in$B) and (x$\in$A and y$\in$C)\\
$\therefore$ we have (x,y)$\in$(A$\times$B) and (x,y)$\in$(A$\times$C)\\
$\therefore$ we have (x,y)$\in$ (A$\times$B) $\cap$ (A$\times$C)\\
$\therefore$  A$\times$(B$\cap$C) $\Subset$ (A$\times$B) $\cap$ (A$\times$C)\\
(2)for every (x,y)$\in$ A$\times$(B$\cap$C), we have (x,y)$\in$(A$\times$B) and (x,y)$\in$(A$\times$C)\\
$\therefore$ we have (x$\in$A and y$\in$B) and (x$\in$A and y$\in$C)\\
$\therefore$ we have x$\in$A and y$\in$ B$\cap$C\\
$\therefore$ we have (x,y)$\in$ A$\times$(B$\cap$C)\\
$\therefore$ (A$\times$B) $\cap$ (A$\times$C) $\Subset$ A$\times$(B$\cap$C)\\
Based on (1) and (2),A$\times$(B$\cap$C) = (A$\times$B) $\cap$ (A$\times$C)\\
\end{solution}






\begin{problem}[UD:9.16]
This problem introduces rigorous definitions of an ordered pair and Cartesian product. Let A be a set and a,b$\in$A. We define the ordered pair of a and b with first coordinate a and second coordinate b as\\
\indent \indent (a,b)=\{\{a\},\{a,b\}\}.\\
Using this definition prove the following.\\
(a) If (a,b)=(x,y), then a=x and b=y.\\
(b) If a$\in$A and b$\in$B, then (a,b) $\in$ P(P(A$\cup$B)).\\
Now we are able to define the Cartesian product of the two sets A and B as the set\\
A$\times$B=\{x$\in$P(P(A$\cup$B))\:x=(a,b) for some a$\in$ A and some b$\in$B\}.\\
(c) Use the above definations to prove that if A$\Subset$C and B$\Subset$D,then A$\times$B $\Subset$ C$\times$D.\\
\indent This is a pretty complicated defination. It is also not our idea, but rather an idea that was born from axioms. P.Halmos' book,[31],is an excellent reference for this subject.

\end{problem}


\begin{solution}
(a)\\
$\because$ (a,b)=(x,y)\\
$\therefore$ \{\{a\},\{a,b\}\}=\{\{x\},\{x,y\}\}\\
$\therefore$ \{a\}==\{x\} and \{a,b\}=\{x,y\}\\
$\therefore$ a=x and b=y\\
(b)\\
$\because$ a$\in$A and b$\in$B\\
$\therefore$ a$\in$A$\cup$B and b$\in$A$\cup$B\\
$\therefore$ {a}$\in$ P(A$\cup$B) and {a,b}$\in$ P(A$\cup$B)\\
$\therefore$ \{\{a\},\{a,b\}\}$\in$P(P(A$\cup$B))\\
$\therefore$ (a,b) $\in$ P(P(A$\cup$B))\\
(c)\\
(1) Prove if a=x and b=y, then (a,b)=(x,y)\\
$\because$ a=x and b=y\\
$\therefore$ \{a\}==\{x\} and \{a,b\}=\{x,y\}\\
$\therefore$ \{\{a\},\{a,b\}\}=\{\{x\},\{x,y\}\}\\
$\therefore$ (a,b)=(x,y)\\
(2)$\because$ A$\Subset$C and B$\Subset$D\\
$\therefore$ for every x$\in$A, we have x$\in$C\\
\indent for every y$\in$B, we have y$\in$D\\
$\therefore$ for every (x,y) $\in$ (A$\times$B),we have (x,y) $\in$ (C$\times$D)\\
$\therefore$ A$\times$B $\Subset$ C$\times$D 

\end{solution}



%%%%%%%%%%
%%%%%%%%%%%%%%%%%%%%
\begincorrection	% begin the ``correction'' part (Optional)

%%%%%%%%%%
\begin{problem}[题号]
  题目。
\end{problem}

\begin{cause}
  简述错误原因(可选)。
\end{cause}

% Or use the ``solution'' environment
\begin{revision}
  正确解答。
\end{revision}
%%%%%%%%%%
%%%%%%%%%%%%%%%%%%%%
\beginfb	% begin the feedback section (Optional)

你可以写:
\begin{itemize}
  \item 对课程及教师的建议与意见
  \item 教材中不理解的内容
  \item 希望深入了解的内容
  \item 等
\end{itemize}
%%%%%%%%%%%%%%%%%%%%
\end{document}