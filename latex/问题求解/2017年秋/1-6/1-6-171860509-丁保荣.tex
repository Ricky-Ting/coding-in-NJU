%%%%%%%%%%%%%%%%%%%%%%%%%%%%%%%%%%%%%%%%%%%%%%%%%%%%%%%%%%%%
% File: hw.tex 						   %
% Description: LaTeX template for homework.                %
%
% Feel free to modify it (mainly the 'preamble' file).     %
% Contact hfwei@nju.edu.cn (Hengfeng Wei) for suggestions. %
%%%%%%%%%%%%%%%%%%%%%%%%%%%%%%%%%%%%%%%%%%%%%%%%%%%%%%%%%%%%

%%%%%%%%%%%%%%%%%%%%%%%%%%%%%%%%%%%%%%%%%%%%%%%%%%%%%%%%%%%%%%%%%%%%%%
% IMPORTANT NOTE: Compile this file using 'XeLaTeX' (not 'PDFLaTeX') %
%
% If you are using TeXLive 2016 on windows,                          %
% you may need to check the following post:                          %
% https://tex.stackexchange.com/q/325278/23098                       %
%%%%%%%%%%%%%%%%%%%%%%%%%%%%%%%%%%%%%%%%%%%%%%%%%%%%%%%%%%%%%%%%%%%%%%

\documentclass[11pt, a4paper, UTF8]{ctexart}
%%%%%%%%%%%%%%%%%%%%%%%%%%%%%%%%%%%
% File: preamble.tex
%%%%%%%%%%%%%%%%%%%%%%%%%%%%%%%%%%%

\usepackage[top = 1.5cm]{geometry}

% Set fonts commands
\newcommand{\song}{\CJKfamily{song}} 
\newcommand{\hei}{\CJKfamily{hei}} 
\newcommand{\kai}{\CJKfamily{kai}} 
\newcommand{\fs}{\CJKfamily{fs}}

\newcommand{\me}[2]{\author{{\bfseries 姓名:}\underline{#1}\hspace{2em}{\bfseries 学号:}\underline{#2}}}

% Always keep this.
\newcommand{\noplagiarism}{
  \begin{center}
    \fbox{\begin{tabular}{@{}c@{}}
      请独立完成作业,不得抄袭。\\
      若参考了其它资料,请给出引用。\\
      鼓励讨论,但需独立书写解题过程。
    \end{tabular}}
  \end{center}
}

% Each hw consists of three parts:
% (1) this homework
\newcommand{\beginthishw}{\part{作业}}
% (2) corrections (Optional)
\newcommand{\begincorrection}{\part{订正}}
% (3) any feedback (Optional)
\newcommand{\beginfb}{\part{反馈}}

% For math
\usepackage{amsmath}
\usepackage{amsfonts}
\usepackage{amssymb}

% Define theorem-like environments
\usepackage[amsmath, thmmarks]{ntheorem}

\theoremstyle{break}
\theorembodyfont{\song}
\theoremseparator{}
\newtheorem*{problem}{题目}

\theorempreskip{2.0\topsep}
\theoremheaderfont{\kai\bfseries}
\theoremseparator{:}
% \newtheorem*{remark}{注}
\theorempostwork{\bigskip\hrule}
\newtheorem*{solution}{解答}
\theorempostwork{\bigskip\hrule}
\newtheorem*{revision}{订正}

\theoremstyle{plain}
\newtheorem*{cause}{错因分析}
\newtheorem*{remark}{注}

\theoremstyle{break}
\theorempostwork{\bigskip\hrule}
\theoremsymbol{\ensuremath{\Box}}
\newtheorem*{proof}{证明}

\renewcommand\figurename{图}
\renewcommand\tablename{表}

% For figures
% for fig with caption: #1: width/size; #2: fig file; #3: fig caption
\newcommand{\fig}[3]{
  \begin{figure}[htp]
    \centering
      \includegraphics[#1]{#2}
      \caption{#3}
  \end{figure}
}

% for fig without caption: #1: width/size; #2: fig file
\newcommand{\fignocaption}[2]{
  \begin{figure}[htp]
    \centering
    \includegraphics[#1]{#2}
  \end{figure}
}  % modify this file if necessary

%%%%%%%%%%%%%%%%%%%%
\title{第六讲:如何将算法告诉计算机}
\me{丁保荣}{171860509}
\date{\today}     % you can specify a date like ``2017年9月30日''.
%%%%%%%%%%%%%%%%%%%%
\begin{document}
\maketitle
%%%%%%%%%%%%%%%%%%%%
\noplagiarism	% always keep this
%%%%%%%%%%%%%%%%%%%%
\beginthishw	% begin ``this homework (hw)'' part

%%%%%%%%%%
\begin{problem}	% NOTE: use '[]' (instead of '()' or '{}') to provide additional information
写出你现在用的 C++ 语言的算术表达式的完整严格的文法
 \end{problem}

\begin{remark}
C++语言运算符来自C++ Primer Plus(第六版) 附录D 运算符优先级\\
英语翻译水平较渣,还请见谅\\
\end{remark}

\begin{solution}
<$\emph{unary-operator}$>::=  $\boldsymbol{!}$|$\boldsymbol{+}$|$\boldsymbol{-}$|$\boldsymbol{++}$|$\boldsymbol{--}$|$\boldsymbol{\&}$|$\boldsymbol{*}$\\
<$\emph{binary-operator}$>::=  $\boldsymbol{*}$| $\boldsymbol{/}$ | $\boldsymbol{+}$ | $\boldsymbol{-}$ \\
<$\emph{scope resolution operator}$>::=  <$\emph{namespace}$> $\boldsymbol{::}$ <$\emph{statement}$>;\\
<$\emph{direct member operator}$>::=  <$\emph{variable}$>$\boldsymbol{.}$<$\emph{variable}$>\\
<$\emph{array-index}$>::=  <$\emph{array-variable}$>$\boldsymbol{[}$<$\emph{value}$>$\boldsymbol{]}$;\\
<$\emph{stepup}$>::=  <$\emph{variable}$>$\boldsymbol{++}$ [;] | $\boldsymbol{++}$<$\emph{variable}$> [;]\\
<$\emph{stepdown}$>::=  <$\emph{variable}$>$\boldsymbol{--}$ [;] | $\boldsymbol{--}$<$\emph{variable}$> [;]\\
<$\emph{negation}$>::=  $\boldsymbol{!}$<$\emph{expression}$>\\
<$\emph{positive-operator}$>::=  $\boldsymbol{+}$<$\emph{value}$>\\
<$\emph{negative-operator}$>::=  $\boldsymbol{-}$<$\emph{value}$>\\
<$\emph{address-access}$>::=  $\boldsymbol{\&}$<$\emph{variable}$>\\
<$\emph{value-access}$>::=  $\boldsymbol{*}$<$\emph{pointer}$>\\
<$\emph{multiply-operator}$>::=  <$\emph{value}$> $\boldsymbol{*}$ <$\emph{value}$>\\
<$\emph{division-operator}$>::=  <$\emph{value}$> $\boldsymbol{/}$ <$\emph{value}$>\\
<$\emph{additive-operator}$>::=  <$\emph{value}$> $\boldsymbol{+}$ <$\emph{value}$>\\
<$\emph{subtraction-operator}$>::=  <$\emph{value}$> $\boldsymbol{-}$ <$\emph{value}$>\\
<$\emph{less-than-expression}$>::=  <$\emph{value}$> $\boldsymbol{<}$ <$\emph{value}$>\\
<$\emph{greater-than-expression}$>::=  <$\emph{value}$> $\boldsymbol{>}$ <$\emph{value}$>\\
<$\emph{equality-expression}$>::=  <$\emph{value}$> $\boldsymbol{==}$ <$\emph{value}$>\\
<$\emph{not-equality-expression}$>::=  <$\emph{value}$> $\boldsymbol{!=}$ <$\emph{value}$>\\
<$\emph{and-operator}$>::=  <$\emph{expression}$> $\boldsymbol{\&\&}$ <$\emph{expression}$>\\
<$\emph{or-expression}$>::=  <$\emph{expression}$> $\boldsymbol{||}$ <$\emph{expression}$>\\
<$\emph{assignment-operator}$>::=  $\boldsymbol{=}$ |$\boldsymbol{*=}$ |$\boldsymbol{/=}$ |$\boldsymbol{\%=}$ |$\boldsymbol{+=}$ |$\boldsymbol{-=}$ |$\boldsymbol{\&=}$ |$\boldsymbol{\^=}$ \\
<combination-operator>::=  <$\emph{expression}$> $\boldsymbol{,}$ <$\emph{expression}$>\\ 

\end{solution}
%%%%%%%%%%

%%%%%%%%%%%%%%%%%%%%
\begincorrection	% begin the ``correction'' part (Optional)

%%%%%%%%%%
\begin{problem}[DH:2.5]
 Show how to perform the following simulations of some control constructs by others. The sequencing construct “and-then” is implicitly available for all the simulations. You may introduce and use new variables and labels if necessary.\\
(a) Simulate a “for-do” loop by a “while-do” loop.\\
(b) Simulate the “if-then” and “if-then-else” statements by “while-do” loops.\\
(c) Simulate a “while-do” loop by “if-then” and “goto” statements.\\
(d) Simulate a “while-do” loop by a “repeat-until” loop and “if-then” statements.
\end{problem}

\begin{cause}
  没有退出while的条件,会造成死循环
\end{cause}

% Or use the ``solution'' environment
\begin{revision}
\begin{verbatim}
(b)
if (A) then
{
	do B;
}
else
{
	do C;
}
 i=0;
 while( (A) && (i==0))
 {
 		do B;
 		i++;
 }
 while ( (not A) && (i==0) )
 {
 		do C;
 		i++;
 }

\end{verbatim}
\end{revision}
%%%%%%%%%%
%%%%%%%%%%%%%%%%%%%%
\beginfb	% begin the feedback section (Optional)

你可以写:
\begin{itemize}
  \item 对课程及教师的建议与意见
  \item 教材中不理解的内容
  \item 希望深入了解的内容
  \item 等
\end{itemize}
%%%%%%%%%%%%%%%%%%%%
\end{document}