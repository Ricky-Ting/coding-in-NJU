%%%%%%%%%%%%%%%%%%%%%%%%%%%%%%%%%%%%%%%%%%%%%%%%%%%%%%%%%%%%
% File: hw.tex 						   %
% Description: LaTeX template for homework.                %
%
% Feel free to modify it (mainly the 'preamble' file).     %
% Contact hfwei@nju.edu.cn (Hengfeng Wei) for suggestions. %
%%%%%%%%%%%%%%%%%%%%%%%%%%%%%%%%%%%%%%%%%%%%%%%%%%%%%%%%%%%%

%%%%%%%%%%%%%%%%%%%%%%%%%%%%%%%%%%%%%%%%%%%%%%%%%%%%%%%%%%%%%%%%%%%%%%
% IMPORTANT NOTE: Compile this file using 'XeLaTeX' (not 'PDFLaTeX') %
%
% If you are using TeXLive 2016 on windows,                          %
% you may need to check the following post:                          %
% https://tex.stackexchange.com/q/325278/23098                       %
%%%%%%%%%%%%%%%%%%%%%%%%%%%%%%%%%%%%%%%%%%%%%%%%%%%%%%%%%%%%%%%%%%%%%%

\documentclass[11pt, a4paper, UTF8]{ctexart}
\input{preamble}  % modify this file if necessary

%%%%%%%%%%%%%%%%%%%%
\title{第九讲:关系及其基本性质}
\me{丁保荣}{171860509}
\date{\today}     % you can specify a date like ``2017年9月30日''.
%%%%%%%%%%%%%%%%%%%%
\begin{document}
\maketitle
%%%%%%%%%%%%%%%%%%%%
\noplagiarism % always keep this
%%%%%%%%%%%%%%%%%%%%
\beginthishw  % begin ``this homework (hw)'' part


\begin{problem}[UD:10.2]
Let X=\{1,2,3,4,5\}.\\
(a) If possible, define a relation on X that is an equivalence relation.\\
(b) If possible, define a relation on X that is reflexive, but neither symmetric nor transitive.\\
(c) If possible, define a relation on X that is symmetric, but neither reflexive nor transitive.\\
(d) If possible, define a relation on X that is transitive, but neither reflexive nor symmetric.\\
\end{problem}
\begin{solution}
(a) define x$\sim$y if and only if x=y\\
(b) define x$\sim$y if and only if 0$\le$x-y$\le$1\\
(c) define x$\sim$y if and only if x=6-y\\
(d) define x$\sim$y :1$\sim$2 , 2$\sim$3 and 1$\sim$3

\end{solution}


\begin{problem}[UD:10.4]
Define a relation $\sim$ on R$^2$ as follows: For (x$_1$,x$_2$),(y$_1$,y$_2$)$\in$ R$^2$, we say that (x$_1$,x$_2$)$\sim$(y$_1$,y$_2$) if and only if both x$_1$-y$_1$ and x$_2$-y$_2$ are even integers. Is this relation an equivalence relation? Why or why not?
\end{problem}
\begin{solution}
Yes, this relation is an equivalence.\\
(i) for reflexive:\\
\indent $\because$  for (x$_1$,x$_2$) $\sim$ (y$_1$,y$_2$), we have both x$_1$-y$_1$ and x$_2$-y$_2$ are even integers\\
\indent $\therefore$  we can let x$_1$-y$_1$ and x$_2$-y$_2$ are 0\\
\indent $\therefore$  we have x$_1$=y$_1$ and x$_2$=y$_2$\\
\indent $\therefore$  we have (x$_1$,x$_2$) $\sim$ (x$_1$,x$_2$)\\
\indent $\therefore$  this relation is reflexive\\
(ii) for symmetric\\
\indent $\because$  for (x$_1$,x$_2$) $\sim$ (y$_1$,y$_2$), we have both x$_1$-y$_1$ and x$_2$-y$_2$ are even integers\\
\indent $\therefore$  we also have both y$_1$-x$_1$ and y$_2$-x$_2$ are even integers\\
\indent $\therefore$  (y$_1$,y$_2$) $\sim$ (x$_1$,x$_2$)\\
\indent $\therefore$  this relation is symmetric\\
(iii)  for transitive\\
\indent $\because$  for (x$_1$,x$_2$) $\sim$ (y$_1$,y$_2$), we have both x$_1$-y$_1$ and x$_2$-y$_2$ are even integers\\
\indent $\therefore$  we also have for (y$_1$,y$_2$) $\sim$ (z$_1$,z$_2$), we have both y$_1$-z$_1$ and y$_2$-z$_2$ are even integers\\
\indent $\therefore$  we have both x$_1$-z$_1$ and x$_2$-z$_2$ are even integers\\
\indent $\therefore$  (x$_1$,x$_2$) $\sim$ (z$_1$,z$_2$)\\
\indent $\therefore$  this relation is transitive\\
\end{solution}


%%%%此题存疑

\begin{problem}[UD:10.5]
Let X be a nonempty set with an equivalence relation $\sim$ on it. Prove that for all elements x and y in X, the equality E$_x$=E$_y$ holds if and only if x$\sim$y.
\end{problem}
\begin{solution}
(1)\\
$\because$ we have E$_x$=E$_y$\\
$\therefore$ \{y$\in$X:x$\sim$y\}=\{x$\in$X:y$\sim$x\}\\
$\therefore$  x$\sim$y\\
(2)\\
$\because$ we have x$\sim$y and this relation is an equivalence\\
$\therefore$ y$\sim$x\\
$\therefore$ \{y$\in$X:x$\sim$y\}=\{x$\in$X:y$\sim$x\}\\
$\therefore$ E$_x$=E$_y$\\
Based on (1) and (2), we have for all elements x and y in X, the equality E$_x$=E$_y$ holds if and only if x$\sim$y.
\end{solution}


\begin{problem}[UD:10.8]
Recall that a $\boldsymbol{polynomial}$ p over R is an expression of the form p(x)=a$_n$x$^n$ + a$_{n-1}$x$^{n-1}$ + ... + a$_1$x$^1$ + a$_0$ where each a$_j$ $\in$ R and n$\in$ N. The largest integer j such that a$_j$ $\not=$ 0 is the degree of p. We define the degree of the constant polynomial p=0 to be -$\infty$. (A polynomial over R defines a function p:R $\rightarrow$ R.)\\
(a) Define a relation on the set of polynomials by p $\sim$ q if and only if p(0)=q(0). Is this an equivalence relation? If so, what is the equivalence class of the polynomial given by p(x)=x?\\
(b) Define a relation on the set of polynomials by p $\sim$ q if and only the degree of p is the same as the degree of q. Is this an equivalence relation? If so, what is E$_r$ if r(x)=3x+5?\\
(c) Define a relation on the set of polynomials by p $\sim$ q if and only the degree of p is less than or equal to the degree of q. Is this an equivalence relation? If so, what is E$_r$ where r(x)=x$^2$?\\
\end{problem}
\begin{solution}
(a) yes.\\
E$_{p(x)=x}$=\{q(x)=a$_n$x$^n$ + a$_{n-1}$x$^{n-1}$ + ... + a$_1$x$^1$ :a$_j$ $\in$ R and n$\in$ N\}\\
(b) yes.\\
E$_{r(x)=3x+5}$=\{q(x)=ax+b : a$\not=$0\}\\
(c) No\\
\end{solution}


\begin{problem}[UD:11.3]
(a) For each r$\in$R, let A$_r$=\{(x,y,z)$\in$ R$^3$ : x+y+z=r\}. Is this a partition of R$^3$? If so, give a geometric description of the partitioning sets.\\
(b) For each r$\in$R, let A$_r$=\{(x,y,z)$\in$ R$^3$ : x$^2$+y$^2$+z$^2$=r$^2$\}. Is this a partition of R$^3$? If so, give a geometric description of the partitioning sets.\\
\end{problem}
\begin{solution}
(a) yes.\\
Each partitioning set is a plane.\\
(b) yes.\\
when r=0,the partitioning set is a dot.\\
for other r, the partitioning set is a sphere(only the surface).
\end{solution}


\begin{problem}[UD:11.7]
Consider the set P of polynomials with real coefficients. Decide whether or not each of the following collection of sets determines a partition of P. If you decide that it does determine a partition, show it carefully. If you decide that it does not determine a partition, list the part(s) of the definition that is (are) not satisfied and justify your claim with an example. (See Problem 10.8 for more information about polynomials.)\\
(a) For m$\in$ N, let A$_m$ denote the set of polynomials of degree m.\\
(b) For c$\in$ R, let A$_c$ denote the set of polynomials such that p(0)=c.\\
(c) For a polynomial q, let A$_q$ denote the set of all polynomials p such that q is a factor of p; that is, there is a polynomial r such that p=qr.\\
(d) For c$\in$ R, let A$_c$ denote the set of polynomials such that p(c)=0.
\end{problem}
\begin{solution}
(a) this doesn't determine a partition of P.\\
$\because$ the degree of (p(x)=0) is -$\infty$\\
$\therefore$the set of (A$_m$) doesn't include B=\{p(x)=0\}\\
\\
(b) this determines a partition of P.\\
(i) we can easily note that A$_c$ is nonempty\\
(ii) first, we can easily conclude that $\bigcup_{c\in R}$A$_c$$\subseteq$ P\\
then, for every polynomial $\in$ P, we can denote it by p(x)=a$_n$x$^n$ + a$_{n-1}$x$^{n-1}$ + ... + a$_1$x$^1$ + c where each a$_j$ $\in$ R and n$\in$ N.\\
$\therefore$ p(0)=0\\
$\therefore$ p(x) $\in$ A$_c$ for some c $\in$ R\\
$\therefore$ P $\subseteq$ $\bigcup_{c\in R}$A$_c$\\
$\therefore$ $\bigcup_{c\in R}$A$_c$ =P\\
(iii) If A$_a$ $\cap$ A$_b$ $\not=$ $\emptyset$ (a,b$\in$R)\\
$\therefore$ there exist a p(x)$\in$ A$_a$ $\cap$ A$_b$\\
$\therefore$ p(0)=a and p(0)=b\\
$\therefore$ a=b\\
$\therefore$ A$_a$=A$_b$\\
Based on (i),(ii) and (iii), this determines a partition of P\\
\\
(c) this doesn't determine a partition of P.\\
A$_{1}$=P while A$_{x}$ $\subsetneqq$ P and A$_{x}$ $\not=$ $\emptyset$\\
$\therefore$ A$_{1}$ $\cap$ A$_{x}$ $\not=$ $\emptyset$ and A$_{1}$$\not=$A$_{x}$\\
$\therefore$ this doesn't satisfy the third condition\\
$\therefore$ this doesn't determine a partition of P.\\
\\
(d) this doesn't determine a partition of P.\\
for p(x)=x$^2$+1, we cannot find c$\in$R to let p(c)=0\\
$\therefore$ p(x)=x$^2$+1 $\notin$ A$_c$ for any of c$\in$R\\
$\therefore$ $\bigcup_{c\in R}$A$_c$ $\not=$P\\
$\therefore$ this doesn't determine a partition of P.\\
\end{solution}



\begin{problem}[UD:11.8]
For two nonempty disjoint sets, I and J, let \{A$_\alpha$: $\alpha \in$ I\} be a partition of R$^+$ and \{A$_\alpha$: $\alpha \in$ J\} be a partition of R$^-$ $\cup$ \{0\}. Prove that \{A$_\alpha$ : $\alpha \in$ I$\cup$J \} is a partition of R.
\end{problem}
\begin{solution}
(i) we can easily note that for every $\alpha \in I \cup J$, A$\alpha$ $\not=$ $\emptyset$\\
(ii) \\
$\because$ \{A$_\alpha$: $\alpha \in$ I\} is a partition of R$^+$ and \{A$_\alpha$: $\alpha \in$ J\} is a partition of R$^-$ $\cup$ \{0\}\\
$\therefore$ $\bigcup_{\alpha \in I}$A$\alpha$ = R$^+$ and $\bigcup_{\alpha \in J}$A$\alpha$ = R$^-$ $\cup$ \{0\}\\
$\therefore$ $\bigcup_{\alpha \in I}$A$\alpha$ $\cup$ $\bigcup_{\alpha \in J}$A$\alpha$ = R$^+$ $\cup$ (R$^-$ $\cup$ \{0\})\\
$\therefore$ $\bigcup_{\alpha \in I\cup J}$A$\alpha$ = R\\
(iii)\\
if  both i and j $\in$ I(or J)  and A$_i$$\cap$A$_j$ $\not=$ $\emptyset$, we conclude from the condition that A$_i$=A$_j$\\
if i$\in$ I and j$\in$ J we can conclude from the condition that A$_i$$\cap$A$_j$ = $\emptyset$\\
$\therefore$ for i,j$\in$ I$\cup$J, if A$_i$$\cap$A$_j$ $\not=$ $\emptyset$, we can conclude that A$_i$=A$_j$\\
Based on (i), (ii) and (iii), \{A$_\alpha$ : $\alpha \in$ I$\cup$J \} is a partition of R.
\end{solution}



\begin{problem}[UD:11.9]
Let X be a nonempty set and \{A$_\alpha$ : $\alpha \in$ I\} be a partition of X.\\
(a) Let B be a subset of X such that A$_\alpha \cap$ B $\not=$ $\emptyset$ for every $\alpha \in$ I. Is \{A$_\alpha \cap B$: $\alpha \in$ I\} a partition of B? Prove it or give a counterexample.\\
(b) Suppose further that A$_\alpha$ $\not=$ X for every $\alpha \in$ I. Is \{X$\backslash$ A$_\alpha$: $\alpha \in$ I\} a partition of X? Prove it or give a counterexample.
\end{problem}
\begin{solution}
(a) yes, \{A$_\alpha \cap B$: $\alpha \in$ I\} is a partition of B\\
(i) we can easily see from the condition that A$_\alpha \cap$ B $\not=$ $\emptyset$ for every $\alpha \in$ I\\
(ii)\\
\indent first, for every $\alpha \in$ I, A$_\alpha \cap$ B $\subseteq$ B \\
$\therefore$ $\bigcup_{\alpha \in I}$ (A$_\alpha \cap$ B) $\subseteq$ B\\
\indent then, $\because$ B$\subseteq$ X and  X=$\bigcup_{\alpha \in I}$ A$_\alpha$ \\
$\therefore$ B$\subseteq$ $\bigcup_{\alpha \in I}$ A$_\alpha$\\
$\therefore$ B$\subseteq$ $\bigcup_{\alpha \in I}$ (A$_\alpha \cap$ B)\\
$\therefore$ $\bigcup_{\alpha \in I}$ (A$_\alpha \cap$ B) = B\\
(iii)\\
if A$_i \cap B$ $\cap$ A$_j \cap B$ $\not=$ $\emptyset$ (i,j$\in$ I)\\
$\therefore$  A$_i$ $\cap$ A$_j$ $\not=$ $\emptyset$ (i,j$\in$ I)\\
$\therefore$ acoodint to the condition , A$_i$=A$_j$\\
$\therefore$ A$_i \cap B$ = A$_j \cap B$\\
Based on (i), (ii) and (iii), \{A$_\alpha \cap B$: $\alpha \in$ I\} is a partition of B\\
(b) No\\
Let X=\{1,2,3\} A$_1$=\{1\}  A$_2$=\{2\}  A$_3$=\{3\} \\
$\therefore$ we have X$\backslash$ A$_1$=\{2,3\} ,X$\backslash$ A$_2$=\{1,3\} ,X$\backslash$ A$_3$=\{1,2\} \\
$\therefore$ A$_1$ $\cap$ A$_2$ =\{3\} $\not=$ $\emptyset$ but A$_1$ $\not=$ A$_2$\\
$\therefore$ it doesn't satisfy the third condition.\\
\end{solution}


\begin{problem}[UD:12.10]
Let S and T be nonempty bounded subsets of R.\\
(a) Show that sup(S$\cup$T) $\ge$ sup S, and  sup(S$\cup$T) $\ge$ sup T.\\
(b) Show that sup(S$\cup$T) = max\{sup S, sup T\}.\\
(c) Try to state the results of (a) and (b) in English, without using mathematical symbols.
\end{problem}
\begin{solution}
(a) \\
$\therefore$ for all x $\in$ S,we have x$\le$sup S and for all y $\in$ T,we have y$\le$sup T\\
there are three cases: S>T,S<T,S=T\\
\indent (i) S>T\\
\indent $\therefore$ sup S$\ge$ x and sup S$\ge$ sup T $\ge$ y\\
\indent $\therefore$ sup S is the upper bound of S$\cup$T\\
\indent if there is M<sup S that M is also the upper bound of S$\cup$T, then M is also the upper bound of S\\
\indent $\therefore$ it contradicts with that sup S is the least upper bound of S\\
\indent $\therefore$ sup S is the supremum of S$\cup$T\\
\indent (ii)S<T\\
\indent it is similar to (i)\\
\indent $\therefore$ sup T is the supremum of S$\cup$T\\
\indent (iii)S=T\\
\indent following the above, we can easily conclude that sup T or sup S is the supremum of S$\cup$T\\
$\therefore$ we can conclude that sup\{S$\cup$T)=max(sup S, sup T)\}\\
$\therefore$ sup(S$\cup$T) $\ge$ sup S, and  sup(S$\cup$T) $\ge$ sup T.\\
(b)\\
I have done it in (a)\\
$\therefore$ we can conclude that sup\{S$\cup$T)=max(sup S, sup T)\}\\
(c)\\
for (a): the supremum of the union of sets S and T is greater than the supremum of set S or that of set T\\
for (b): the supremum of the union of sets S and T is the maximum of the supremum of set S and that of set T\\

\end{solution}



\begin{problem}[UD:12.13b]
Let $\sim$ denote a relation on a set S. The relation $\sim$ is called a $\boldsymbol{partial order}$ if the following three conditions are satisfied.\\
(i) (Reflexive property) For all x$\in$ S, we have x$\sim$x.\\
(ii) (Transitive property) For all x,y,z$\in$S, if x$\sim$y and y$\sim$z,then x$\sim$z.\\
(iii)(Antisymmetric property) For all x,y$\in$ S, if x$\sim$y and y$\sim$x, then x=y.\\
The relation $\sim$ is a $\boldsymbol{total order}$ on the set S if, in addition, (iv) below is satisfied.\\
(iv) For all x,y$\in$S, either x$\sim$y or y$\sim$x.\\
(b) Let A be a set containing at least two elements. We define an order on P(A) using the regular set inclusion $\Subset$. Show that (P(A),$\Subset$) is a partial order, but not a total order.
\end{problem}
\begin{solution}
(b)\\
(i) for every x$\in$ $\mathcal{P}$(A), we can easily conclude that x$\subseteq$x\\
$\therefore$ x$\sim$x\\
(ii) for every x,y,z$\in$$\mathcal{P}$(A) and x$\sim$y and y$\sim$z, we have x$\subseteq$y and y$\subseteq$z\\
$\therefore$ x$\subseteq$z\\
$\therefore$ x$\sim$z\\
(iii) for every x,y$\in$$\mathcal{P}$(A) and x$\sim$y and y$\sim$x, we have x$\subseteq$y and y$\subseteq$x\\
$\therefore$ x=y\\
(iv) for z,w$\in$A, let x=\{z\}, y=\{w\}, we have x,y$\in$ $\mathcal{P}$(A), but x$\nsubseteq$y and y$\nsubseteq$x\\
Based on (i), (ii), (iii), and (iv) ,we can conclude that (P(A),$\Subset$) is a partial order, but not a total order.\\
\end{solution}


\begin{problem}[UD:12.16]
You showed in Problem 12.13 that (P(Z),$\Subset$) is a partial order. For every nonempty subset $\mathcal{A}$ of $\mathcal{P(Z)}$ we say that U$\in$ $\mathcal{P(Z)}$ is an upper set of $\mathcal{A}$, if X$\Subset$U for all X$\in$ $\mathcal{A}$. A nonempty set $\mathcal{A}$ $\Subset$ $\mathcal{P(Z)}$ will be called an upper bounded set if there is an upper set of $\mathcal{A}$ in $\mathcal{P(Z)}$. We say U$_0$ $\in$ $\mathcal{P(Z)}$ is a least upper set if (i) U$_0$ is an upper set of $\mathcal{A}$ and (ii) if U is another upper set of $\mathcal{A}$, then $U_0$ $\Subset$ U.\\
(a) Let $\mathcal{B}$=\{\{ 1,2,5,7\},\{2,8,10\},\{2,5,8\}\}. Show that $\mathcal{B}$ is an upper bounded set and find a least upper set of $\mathcal{B}$, if there is one.\\
(b) Prove that every nonempty subset of $\mathcal{P(Z)}$ is upper bounded.\\
(c) Define "lower set," "lower bounded set," and "greatest lower set."\\
(d) Let $\mathcal{A}$ be a nonempty subset of $\mathcal{P(Z)}$. Using union and intersection, find an expression for least upper set of $\mathcal{A}$ and greatest lower set of $\mathcal{A}$.\\
(e) Prove that ($\mathcal{P(Z)}$,$\Subset$) has the "least upper set property" (in other words, show every upper bounded set has a least upper set).
\end{problem}
\begin{solution}
(a) Let U$_0$=\{1,2,5,7,8,10\}\\
$\therefore$ for every X$\in$ $\mathcal{B}$, we X$\subseteq$U$_0$\\
$\therefore$ $\mathcal{B}$ is an upper bounded set\\
And the least upper set of $\mathcal{B}$ is U$_0$=\{1,2,5,7,8,10\}\\
(b) Let U=$\mathcal{Z}$\\
$\therefore$ for every $\mathcal{A}$ $\subseteq$ $\mathcal{P(Z)}$ and every X$\in$ $\mathcal{A}$ , we have X $\subseteq$ $\mathcal{Z}$\\
$\therefore$ X $\subseteq$ U\\
$\therefore$ $\mathcal{A}$ is an upper bounded set\\
$\therefore$ every nonempty subset of $\mathcal{P(Z)}$ is upper bounded.\\
(c) \\
For every nonempty subset $\mathcal{A}$ of $\mathcal{P(Z)}$ we say that U$\in$ $\mathcal{P(Z)}$ is a lower set of $\mathcal{A}$, if U$\subseteq$X for all X$\in$ $\mathcal{A}$. A nonempty set $\mathcal{A}$ $\subseteq$ $\mathcal{P(Z)}$ will be called an lower bounded set if there is a lower set of $\mathcal{A}$ in $\mathcal{P(Z)}$. We say U$_0$ $\in$ $\mathcal{P(Z)}$ is a greatest upper set if (i) U$_0$ is a lower set of $\mathcal{A}$ and (ii) if U is another lower set of $\mathcal{A}$, then U $\subseteq$ U$_0$.\\
(d)\\
the least upper set of $\mathcal{A}$=$\bigcup$A$_\alpha$ (for every A$\alpha$$\in$$\mathcal{A}$)\\
the greatest lower set of $\mathcal{A}$=$\bigcap$A$_\alpha$ (for every A$\alpha$$\in$$\mathcal{A}$)\\
(e)\\
for every upper bounded set $\mathcal{A}$, we can let U$_0$=$\bigcup$A$_\alpha$ (for every A$\alpha$$\in$$\mathcal{A}$)\\
$\therefore$ we can easily conclude that for every X$\in$ $\mathcal{A}$, we have X$\subseteq$ U$_0$\\
$\therefore$ U$_0$ is an upper set of $\mathcal{A}$\\
Then ,we need to prove U$_0$is the least one\\
we set a set U$_1$ $\subsetneq$ U$_0$\\
there is x$\notin$ U$_1$ but $\in$ $\bigcup$A$_\alpha$ (for every A$\alpha$$\in$$\mathcal{A}$)\\
then there is x$\notin$ U$_1$ but $\in$ A$_\alpha$ (A$\alpha$$\in$$\mathcal{A}$)\\
we can let X=A$\alpha$ , then X$\nsubseteq$  U$_1$\\
$\therefore$ U$_1$ is not an upper set of $\mathcal{A}$\\
$\therefore$ U$_0$is the least upper set of $\mathcal{A}$\\
$\therefore$ every upper bounded set has a least upper set\\
and we prove that every nonempty subset of $\mathcal{P(Z)}$ is upper bounded in (b)\\
$\therefore$ ($\mathcal{P(Z)}$,$\Subset$) has the "least upper set property"

\end{solution}


\begin{problem}[UD:12.20]
Suppose we define $\infty$ to be an object that satisfies a$\le$ $\infty$ for all a$\in$ R. Prove that $\infty$ $\not=$ R.
\end{problem}
\begin{solution}
Let we suppose $\infty$ $\in$ R\\
$\therefore$ $\infty$+1 $\in$ R\\
$\therefore$ $\infty$+1 > $\infty$ \\
but it contradicts with a$\le$ $\infty$ for all a$\in$ R\\
$\therefore$ what we suppose is not right\\
$\therefore$ $\infty$ $\not=$ R.
\end{solution}


\begin{problem}[UD:12.22]
Prove that if a is a rational number, then there is an irrational number b such that a<b.
\end{problem}
\begin{solution}
let b=a+$\sqrt{2}$\\
$\because$ $\sqrt{2}$ is an irrational number and a is a rational number\\
$\therefore$ b=a+$\sqrt{2}$ is am irrational number and b>a\\
$\therefore$ if a is a rational number, then there is an irrational number b such that a<b.\\
\end{solution}



\begin{problem}[UD:12.23]
Prove that for two arbitrary real numbers a and b with a<b, there is an irrational number c such that a<c<b.(Hint: Consider a/$\sqrt{2}$ and b/$\sqrt{2}$.)
\end{problem}
\begin{solution}
we can find an x that a/$\sqrt{2}$<x<b/$\sqrt{2}$\\
$\therefore$ according to Theorem 12.11 we have find an x that is a rational number\\
let x=$\sqrt{2}$c\\
$\therefore$ c is an irrational number and a<c<b\\
$\therefore$ for two arbitrary real numbers a and b with a<b, there is an irrational number c such that a<c<b.
  
\end{solution}







%%%%%%%%%%%%%%%%%%%%
\begincorrection	% begin the ``correction'' part (Optional)

%%%%%%%%%%
\begin{problem}[题号]
  题目。
\end{problem}

\begin{cause}
  简述错误原因(可选)。
\end{cause}

% Or use the ``solution'' environment
\begin{revision}
  正确解答。
\end{revision}
%%%%%%%%%%
%%%%%%%%%%%%%%%%%%%%
\beginfb	% begin the feedback section (Optional)

你可以写:
\begin{itemize}
  \item 对课程及教师的建议与意见
  \item 教材中不理解的内容
  \item 希望深入了解的内容
  \item 等
\end{itemize}
%%%%%%%%%%%%%%%%%%%%
\end{document}