%%%%%%%%%%%%%%%%%%%%%%%%%%%%%%%%%%%%%%%%%%%%%%%%%%%%%%%%%%%%
% File: hw.tex 						   %
% Description: LaTeX template for homework.                %
%
% Feel free to modify it (mainly the 'preamble' file).     %
% Contact hfwei@nju.edu.cn (Hengfeng Wei) for suggestions. %
%%%%%%%%%%%%%%%%%%%%%%%%%%%%%%%%%%%%%%%%%%%%%%%%%%%%%%%%%%%%

%%%%%%%%%%%%%%%%%%%%%%%%%%%%%%%%%%%%%%%%%%%%%%%%%%%%%%%%%%%%%%%%%%%%%%
% IMPORTANT NOTE: Compile this file using 'XeLaTeX' (not 'PDFLaTeX') %
%
% If you are using TeXLive 2016 on windows,                          %
% you may need to check the following post:                          %
% https://tex.stackexchange.com/q/325278/23098                       %
%%%%%%%%%%%%%%%%%%%%%%%%%%%%%%%%%%%%%%%%%%%%%%%%%%%%%%%%%%%%%%%%%%%%%%

\documentclass[11pt, a4paper, UTF8]{ctexart}
%%%%%%%%%%%%%%%%%%%%%%%%%%%%%%%%%%%
% File: preamble.tex
%%%%%%%%%%%%%%%%%%%%%%%%%%%%%%%%%%%

\usepackage[top = 1.5cm]{geometry}

% Set fonts commands
\newcommand{\song}{\CJKfamily{song}} 
\newcommand{\hei}{\CJKfamily{hei}} 
\newcommand{\kai}{\CJKfamily{kai}} 
\newcommand{\fs}{\CJKfamily{fs}}

\newcommand{\me}[2]{\author{{\bfseries 姓名:}\underline{#1}\hspace{2em}{\bfseries 学号:}\underline{#2}}}

% Always keep this.
\newcommand{\noplagiarism}{
  \begin{center}
    \fbox{\begin{tabular}{@{}c@{}}
      请独立完成作业,不得抄袭。\\
      若参考了其它资料,请给出引用。\\
      鼓励讨论,但需独立书写解题过程。
    \end{tabular}}
  \end{center}
}

% Each hw consists of three parts:
% (1) this homework
\newcommand{\beginthishw}{\part{作业}}
% (2) corrections (Optional)
\newcommand{\begincorrection}{\part{订正}}
% (3) any feedback (Optional)
\newcommand{\beginfb}{\part{反馈}}

% For math
\usepackage{amsmath}
\usepackage{amsfonts}
\usepackage{amssymb}

% Define theorem-like environments
\usepackage[amsmath, thmmarks]{ntheorem}

\theoremstyle{break}
\theorembodyfont{\song}
\theoremseparator{}
\newtheorem*{problem}{题目}

\theorempreskip{2.0\topsep}
\theoremheaderfont{\kai\bfseries}
\theoremseparator{:}
% \newtheorem*{remark}{注}
\theorempostwork{\bigskip\hrule}
\newtheorem*{solution}{解答}
\theorempostwork{\bigskip\hrule}
\newtheorem*{revision}{订正}

\theoremstyle{plain}
\newtheorem*{cause}{错因分析}
\newtheorem*{remark}{注}

\theoremstyle{break}
\theorempostwork{\bigskip\hrule}
\theoremsymbol{\ensuremath{\Box}}
\newtheorem*{proof}{证明}

\renewcommand\figurename{图}
\renewcommand\tablename{表}

% For figures
% for fig with caption: #1: width/size; #2: fig file; #3: fig caption
\newcommand{\fig}[3]{
  \begin{figure}[htp]
    \centering
      \includegraphics[#1]{#2}
      \caption{#3}
  \end{figure}
}

% for fig without caption: #1: width/size; #2: fig file
\newcommand{\fignocaption}[2]{
  \begin{figure}[htp]
    \centering
    \includegraphics[#1]{#2}
  \end{figure}
}  % modify this file if necessary
\usepackage{listings}
\lstloadlanguages{C++, csh, make}
\lstset{language=C++,tabsize=4, keepspaces=true,
	breakindent=22pt, 
	numbers=left,stepnumber=1,numberstyle=\tiny,
	basicstyle=\footnotesize, 
	showspaces=false,
	flexiblecolumns=true,
	breaklines=true, breakautoindent=true,breakindent=4em,
	escapeinside={/*@}{@*/}
}
%%%%%%%%%%%%%%%%%%%%
\title{绪论:计算思维}
\me{丁保荣}{171860509}
\date{\today}     % you can specify a date like ``2017年9月30日''.
%%%%%%%%%%%%%%%%%%%%
\begin{document}
\maketitle
%%%%%%%%%%%%%%%%%%%%
\noplagiarism	% always keep this
%%%%%%%%%%%%%%%%%%%%
\beginthishw	% begin ``this homework (hw)'' part

%%%%%%%%%%
\begin{problem}[1]	% NOTE: use '[]' (instead of '()' or '{}') to provide additional information
1,请你设计一个递归程序:程序输入为n个硬币,第m个为假币.程序输出寻找假币的过程和称量次数。
\end{problem}


\begin{solution}
\begin{lstlisting}

#include <iostream>
//假设假币较轻
using namespace std;
void find(int pt[],int t,int start,int num); //寻找的递归函数
int weigh(int pt[],int t,int start);  //比较重量(计算出重量用来比较大小)
int main(void)
{
    int n,m;  //n是总的硬币数。 m是假币所在的位次。
    cout <<"Please enter the number of coins and the place of the faked one"<<endl;
    cin >> n >> m;
    int pt[10000];   //pt是存放硬币的空间
    for(int i=1;i<=n;i++)
    {
        pt[i]=1;
    }
    
    pt[m]=0;     //将假币重量设为0,其他设为1
    find(pt,n,1,0);
    return 0;
}

void find(int pt[],int t,int start ,int num)
{
    if(t==1)
    {
        cout<<"the faked coin is at"<<start<<endl<<"次数是"<<num<<endl;
    }
    else if(t==2)
    {
        if(pt[start]>pt[start+1])
            cout<<"the faked coin is at "<<start+1 <<endl <<"次数是"<<num+1<<endl;
        else
            cout<<"the faked coin is at "<<start <<endl <<"次数是"<<num+1<<endl;
    }        //以上两个if用来判断递归是否到底,并输出相应结果
    else
    {
        int t1,t2,t3;
        if ((t%3)==0)
            t1=t2=t3=t/3;
        else
        {
            t1=t2=t/3;
            t3=t-t1-t2;
        }
        //将硬币分为基本相等的三堆
        
        if(weigh(pt,t1,start)==weigh(pt,t2,start+t1))
        {
            cout<<"假币在"<<start+t1+t2<<"和"<<start+t1+t2+t3<<"之间"<<endl;
            find(pt,t3,start+t1+t2,num+1);
        }
        else if(weigh(pt,t1,start)>weigh(pt,t2,start+t1))
        {
            cout<<"假币在"<<start+t1<<"和"<<start+t1+t2<<"之间"<<endl;

            find(pt,t2,start+t1,num+1);
        }
        else
        {
            cout<<"假币在"<<start<<"和"<<start+t1<<"之间"<<endl;
            find(pt,t1,start,num+1);
        }

        //判断硬币轻重,区分出假币在哪一堆
    }
}

int weigh(int pt[],int t,int start)
{
    int sum=0;
    for(int i=start;i<=start+t-1;i++)
        sum+=pt[i];   //将所求空间内硬币的重量相加
    return sum;
}

\end{lstlisting}
\end{solution}
%%%%%%%%%%

%%%%%%%%%%
\begin{problem}[2]
2,请你为某个型号的电子词典,设计一个查找单词的递归算法(伪代码)
提示:1,电子词典已经按照词典序排好了序,,词典中共有n个单词;
           2,你可以直接使用compare(x,y)函数来判断单词x和y是否相同;compare函数在单词x排在y之前时,得到值-1,相同时得到值0,之后时得到值1;
           3,请自行查阅“折半查找法”,并从中获得帮助;
           4,查找的结果是:“没有发现”或者“发现”
\end{problem}

\begin{solution}
\begin{lstlisting}

#include <iostream>
#include <string>
using namespace std;
struct stringarray
{
    string word;
}; //定义字典结构
void search(int start,int end,string word,stringarray dictionary[]); //查找函数
int main(void)
{
    stringarray dictionary[10000]; //声明一个字典变量
    string word; //待输入的单词
    int num; //字典内的单词数
    cout<<"Please enter the sum of the words in the dictionary"<<endl;
    cin >> num;
    cout<<"Please enter the words in the dictionary,each line only one word"<<endl;
    cin.get();
    for(int i=1;i<=num;i++)
        getline(cin,dictionary[i].word);  //读入字典
    cout<<"Please enter a word"<<endl;
    cin>> word;     //读入待查单词
    search(1,num,word,dictionary);  //查找
    return 0;
}
void search(int start,int end,string word,stringarray dictionary[])
{
    int mid=(start+end)/2;
    if((end-start)<=1&&(dictionary[start].word!=word)&&(dictionary[end].word!=word))
        cout<<"没有发现";
    else if(dictionary[mid].word==word)
        cout<<"发现";
    else if (dictionary[mid].word>word)
        search(start,mid,word,dictionary);
    else
        search(mid,end,word,dictionary);
    
}





\end{lstlisting}
\end{solution}
%%%%%%%%%%
% \newpage  % continue in a new page
%%%%%%%%%%
%%%%%%%%%%
%%%%%%%%%%%%%%%%%%%%
\begincorrection	% begin the ``correction'' part (Optional)

%%%%%%%%%%
\begin{problem}[题号]
  题目。
\end{problem}

\begin{cause}
  简述错误原因(可选)。
\end{cause}

% Or use the ``solution'' environment
\begin{revision}
  正确解答。
\end{revision}
%%%%%%%%%%
%%%%%%%%%%%%%%%%%%%%
\beginfb	% begin the feedback section (Optional)

你可以写:
\begin{itemize}
  \item 对课程及教师的建议与意见
  \item 教材中不理解的内容
  \item 希望深入了解的内容
  \item 等
\end{itemize}
%%%%%%%%%%%%%%%%%%%%
\end{document}